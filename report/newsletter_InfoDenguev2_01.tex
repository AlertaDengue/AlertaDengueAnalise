%%%%%%%%%%%%%%%%%%%%%%%%%%%%%%%%%%%%%%%%%
% Boletim semanal estadual do InfoDengue
% Adapted for InfoDengue by Claudia Codeco and Thais Riback (May 2016)
%
% Created by:
% Bob Kerstetter (https://www.tug.org/texshowcase/) and extensively modified by:
% Vel (vel@latextemplates.com)
% 
% Original template downloaded from:
% http://www.LaTeXTemplates.com
%
% License:
% CC BY-NC-SA 3.0 (http://creativecommons.org/licenses/by-nc-sa/3.0/)
%%%%%%%%%%%%%%%%%%%%%%%%%%%%%%%%%%%%%%%%%

\documentclass[10pt]{article} % The default font size is 10pt; 11pt and 12pt are alternatives
%%%%%%%%%%%%%%%%%%%%%%%%%%%%%%%%%%%%%%%%%
% Professional Newsletter Template
% Structural Definitions File
% Version 1.0 (09/03/14)
%
% Created by:
% Vel (vel@latextemplates.com)
% 
% This file has been downloaded from:
% http://www.LaTeXTemplates.com
%
% License:
% CC BY-NC-SA 3.0 (http://creativecommons.org/licenses/by-nc-sa/3.0/)
%
%%%%%%%%%%%%%%%%%%%%%%%%%%%%%%%%%%%%%%%%%

%----------------------------------------------------------------------------------------
%	REQUIRED PACKAGES
%----------------------------------------------------------------------------------------

\usepackage{graphicx} % Required for including images
\usepackage[utf8]{inputenc}
\usepackage{microtype} % Improved typography
\usepackage{multicol} % Used for the two-column layout of the document
\usepackage{booktabs} % Required for nice horizontal rules in tables
\usepackage{wrapfig} % Required for in-line images
\usepackage{float} % Required for forcing figures not to float with the [H] 
\usepackage{longtable}
\usepackage{fancyhdr} % footer 
%parameter
\usepackage{Sweave}

%-------------

% -------------
% footer style
\pagestyle{fancy}
\fancyhead{}
\fancyfoot{}
\renewcommand{\headrulewidth}{0.0pt}
\renewcommand{\footrulewidth}{0.4pt}

%------------------------------------------------
% Fonts

\usepackage{charter} % Use the Charter font as the main document font
\usepackage{courier} % Use the Courier font for \texttt (monospaced) only
%\usepackage[T1]{fontenc} % Use T1 font encoding
\usepackage[utf8]{inputenc}
\usepackage[brazilian]{babel}

%------------------------------------------------
% List Separation

\usepackage{enumitem} % Required to customize the list environments
\setlist{noitemsep,nolistsep} % Remove spacing before, after and within lists for a compact look

%------------------------------------------------
% Figure and Table Caption Styles

\usepackage{caption} % Required for changing caption styles
\captionsetup[table]{labelfont={bf,sf},labelsep=period,justification=justified} % Specify the table caption style
\captionsetup[figure]{labelfont={sf,bf},labelsep=period,justification=justified, font=small} % Specify the figure caption style
\setlength{\abovecaptionskip}{10pt} % Whitespace above captions

%------------------------------------------------
% Spacing Between Paragraphs

\makeatletter
\usepackage{parskip}
\setlength{\parskip}{6pt}
\newcommand{\@minipagerestore}{\setlength{\parskip}{6pt}}
\makeatother

%----------------------------------------------------------------------------------------
%	PAGE MARGINS AND SPACINGS
%----------------------------------------------------------------------------------------

\textwidth = 7 in % Text width
\textheight = 9.5 in % Text height
\oddsidemargin = -10pt % Left side margin on odd pages
\evensidemargin = -10pt % Left side margin on even pages
\topmargin = -10pt % Top margin
\headheight = 0pt % Remove the header by setting its space to 0
\headsep = 0pt % Remove the space between the header and top of the page
\parskip = 4pt % Space between paragraph
\parindent = 0.0in % Paragraph indentation
%\pagestyle{empty} % Disable page numbering

%----------------------------------------------------------------------------------------
%	COLORS
%----------------------------------------------------------------------------------------

\usepackage[dvipsnames,svgnames]{xcolor} % Required to specify custom colors
\definecolor{altncolor}{rgb}{.0,0.0,0.8} % Dark blue
\definecolor{myred}{rgb}{.8,0,0} % Dark red
\definecolor{myorange}{rgb}{.93,.57,.0} % Dark orange
\definecolor{myyellow}{rgb}{.93,.93,.0} % yellow
\definecolor{mygreen}{rgb}{.0,.55,.27} % dark green
\definecolor{mywhite}{rgb}{1,1,1} % dark green

\usepackage[colorlinks=true, linkcolor=altncolor, anchorcolor=altncolor, citecolor=altncolor, filecolor=altncolor, menucolor=altncolor, urlcolor=altncolor]{hyperref} % Use the color defined above for all links

% --------------------------------------
% Bullet styles
\usepackage{pifont}
\newcommand{\gsquare}{\item[\color{mygreen}\ding{110}]} 
\newcommand{\ysquare}{\item[\color{myyellow}\ding{110}]} 
\newcommand{\osquare}{\item[\color{myorange}\ding{110}]} 
\newcommand{\rsquare}{\item[\color{myred}\ding{110}]} 
\newcommand{\wsquare}{\item[\color{mywhite}\ding{110}]} 


%----------------------------------------------------------------------------------------
%	BOX STYLES
%----------------------------------------------------------------------------------------

\usepackage[framemethod=TikZ]{mdframed}% Required for creating boxes
\mdfdefinestyle{sidebar}{
    linecolor=black, % Outer line color
    outerlinewidth=0.5pt, % Outer line width
    roundcorner=0pt, % Amount of corner rounding
    innertopmargin=10pt, % Top margin
    innerbottommargin=10pt, % Bottom margin
    innerrightmargin=10pt, % Right margin
    innerleftmargin=10pt, % Left margin
    backgroundcolor=white, % Box background color
    frametitlebackgroundcolor=white, % Title background color
    frametitlerule=false, % Title rule - true or false
    frametitlerulecolor=white, % Title rule color
    frametitlerulewidth=0.5pt, % Title rule width
    frametitlefont=\Large, % Title heading font specification
    font=\small
}

\mdfdefinestyle{intextbox}{
    linecolor=blue, % Outer line color
    outerlinewidth=0.5pt, % Outer line width
    roundcorner=10pt, % Amount of corner rounding
    innertopmargin=7pt, % Top margin
    innerbottommargin=7pt, % Bottom margin
    innerrightmargin=7pt, % Right margin
    innerleftmargin=7pt, % Left margin
    backgroundcolor=white, % Box background color
    frametitlebackgroundcolor=white, % Title background color
    frametitlerule=false, % Title rule - true or false
    frametitlerulecolor=white, % Title rule color
    frametitlerulewidth=0.5pt, % Title rule width
    frametitlefont=\Large % Title heading font specification
}

%----------------------------------------------------------------------------------------
%	HEADING STYLE
%----------------------------------------------------------------------------------------

\newcommand{\heading}[2]{ % Define the \heading command
%\vspace{#2} % White space above the heading
{\begin{center}\Large\textbf{#1}\end{center}} % The heading style
%\vspace{#2} % White space below the heading
}

%\newcommand{\Inicio}{\hyperlink{contents}{{\small Inicio}}} % Define a command for linking back to the contents of the newsletter
\newcommand{\BackToContents}{\hyperlink{contents}{{\small Início}}} % Include the document which specifies all packages and structural customizations for this template

% Carregando dados  (atualmente é manual para cada estado)

\fancyfoot[C]{Boletim Estadual - Rio de Janeiro}
\fancyfoot[R]{Semana 13 de 2016}
\fancyfoot[L]{\href{http://info.dengue.mat.br}{InfoDengue}}

\usepackage{Sweave}
\begin{document}
\Sconcordance{concordance:newsletter_InfoDenguev2_01.tex:newsletter_InfoDenguev2_01.Rnw:%
1 18 1 1 0 17 1 1 8 18 1 1 6 6 0 1 3 61 1 1 60 175 0 1 2 110 1}


%---------------------------------------------------------------------------------
%	HEADER IMAGE
%---------------------------------------------------------------------------------

\begin{figure}[H]
\centering\includegraphics[width=1\linewidth]{InfoDengue2.png}  
\end{figure}

\centerline {\color{altncolor}\rule{1\linewidth}{2.75pt}} % Horizontal line

%---------------------------------------------------------------------------------
%	CAIXA LATERAL - PRIMEIRA PAGINA
%--------------------------------------------------------------------------------

\begin{minipage}[t]{.30\linewidth} % Mini page taking up 30% of the actual page
\begin{mdframed}[style=sidebar,frametitle={}] % Sidebar box

%-----------------------------------------------------------

\hypertarget{contents}{\textbf{{\large Boletim Semanal}}} 

\textbf{Semana 13 de 2016} % se, ano, em pp.RData

\begin{itemize}
\item \hyperlink{estado}{O Estado} 
\end{itemize}


\textbf{Regionais de Saúde:}


\begin{itemize}\item \hyperlink{MtrI}{Metropolitana I}
\item \hyperlink{MtII}{Metropolitana II}
\item \hyperlink{LtrS}{Litoral Sul}
\item \hyperlink{MdPr}{Médio Paraíba}
\item \hyperlink{CntS}{Centro Sul}
\item \hyperlink{Srrn}{Serrana}
\item \hyperlink{BxdL}{Baixada Litorânea}
\item \hyperlink{Nort}{Norte}
\item \hyperlink{Nrst}{Noroeste}\end{itemize}
\textbf{Municípios:}

\begin{itemize}
\item \hyperlink{municips}{Tabela de Municípios} 
\end{itemize}

\centerline {\rule{.75\linewidth}{.25pt}} % Horizontal line

%-----------------------------------------------------------

\hyperlink{vartab}{Variáves nas Tabelas}

\hyperlink{notas}{Notas} % These link to their appropriate sections in the newsletter

\hyperlink{creditos}{Creditos} % These link to their appropriate sections in the newsletter

\centerline {\rule{.75\linewidth}{.25pt}} % Horizontal line

\textbf{Contato}
\begin{description}
\item \href{mailto:alerta\_dengue@fiocruz.br}{alerta\_dengue@fiocruz.br}  
\end{description}

\end{mdframed}
\end{minipage}\hfill % End the sidebar mini page 
\begin{minipage}[t]{.66\linewidth} % Nao pode colocar espaco acima senao ela nao fica lado a lado

%--------------------------------------------------------------------------------
%	TEXTO PRINCIPAL - PRIMEIRA PAGINA - ALERTA A NIVEL DO ESTADO
%-------------------------------------------------------------------------------
\hypertarget{estado}{\heading{Situação da Dengue no Estado do Rio de Janeiro}{6pt}} % \hypertarget provides a label to reference using 

Desde o início do ano, 43785 casos foram registrados no estado, sendo 2763 na última semana. A figura abaixo mostra as condições de transmissão em cada município.

% Mapa estadual
\includegraphics[width=0.8\textwidth]{RJ/figs/Mapa_ERJ.png}
%-----------------------------------------------------------

Dos 92 municipios, 3 encontram-se em nivel verde, 4 em nivel amarelo, 3 em nivel laranja e 3 em nivel vermelho referentes a semana epidemiológica 13-2016. Para informações mais atualizadas
sobre o município do Rio de Janeiro ter acesso ao mapa interativo do estado consultar em \href{http://info.dengue.mat.br}{\textit{Info Dengue}}.


%--------------------------------------------------------------------------------
%	PRIMEIRA PAGINA : BOX - CODIGO DE CORES
%--------------------------------------------------------------------------------
\vspace{1cm}
\begin{mdframed}[style=intextbox,frametitle={}] % Sidebar box

\hypertarget{descriptivebox}{\heading{O código de Cores}{1pt}} % \hypertarget provides a label to reference using \hyperlink{label}{link text}
As cores indicam niveis de atenção
\begin{description}
\item[Verde:] temperaturas amenas, baixa incidência de casos.      
\item[Amarelo:] temperatura propícia para a população do vetor e transmissão da dengue.
\item[Laranja:] transmissão aumentada e sustentada de dengue. 
\item[Vermelho:] incidência alta de dengue, acima dos 90\% históricos.
\end{description}
\end{mdframed}

 \textbf{Na semana passada:} 3 municípios em nivel amarelo, 2 em laranja e 2 em vermelho.    

\end{minipage} % End the main body - first page mini page


%----------------------------------------------------------------------------------
%	SEGUNDA PAGINA: TEXTO PRINCIPAL - LOOP PELAS REGIONAIS
%----------------------------------------------------------------------------------

\begin{minipage}[t]{.66\linewidth}
\hypertarget{MtrI}{\heading{Regional Metropolitana I}}
\includegraphics[width=0.8\textwidth]{RJ/figs/MapaRJ_MetropolitanaI.png}\vspace{0.5cm}\captionof{table}{Resumo das últimas seis semanas epidemiológicas na Regional Metropolitana I }\begin{center}
% latex table generated in R 3.3.1 by xtable 1.8-2 package
% Tue Sep  6 10:49:32 2016
\begin{tabular}{c|ccccccc}
  \hline
SE & temperatura & tweet & casos notif & casos preditos & ICmin & ICmax & incidência \\ 
  \hline
201630 & 17 & 0 & 181 & 200 & 194 & 201 & 2 \\ 
  201631 & 18 & 0 & 98 & 111 & 105 & 112 & 1 \\ 
  201632 & 18 & 0 & 93 & 113 & 107 & 114 & 1 \\ 
  201633 & 19 & 0 & 105 & 146 & 138 & 149 & 1 \\ 
  201634 & 16 & 0 & 70 & 121 & 109 & 124 & 1 \\ 
  201635 & 19 &  & 34 & 92 & 79 & 96 & 0 \\ 
   \hline
\end{tabular}

\end{center}
\vspace{0.5cm}\begin{center}
\captionof{figure}{Casos notificados de dengue e Índice de menção em midia social sobre dengue na Regional Metropolitana I }\includegraphics[width=1.3\textwidth]{RJ/figs/tweetRJ_MetropolitanaI.png}\end{center}
\end{minipage}\hfill\begin{minipage}[t]{.30\linewidth}
\begin{mdframed}[style=sidebar,frametitle={}]
\textbf{Municipios}\begin{itemize}\ysquare Belford Roxo 
\osquare Duque de Caxias 
\ysquare Itaguaí 
\ysquare Japeri 
\wsquare Magé 
\ysquare Mesquita 
\wsquare Nilópolis 
\osquare Nova Iguaçu 
\wsquare Queimados 
\osquare Rio de Janeiro 
\ysquare São João de Meriti 
\wsquare Seropédica 
\end{itemize}\BackToContents\end{mdframed}\hfill\end{minipage}\newpage\begin{minipage}[t]{.66\linewidth}
\hypertarget{MtII}{\heading{Regional Metropolitana II}}
\includegraphics[width=0.8\textwidth]{RJ/figs/MapaRJ_MetropolitanaII.png}\vspace{0.5cm}\captionof{table}{Resumo das últimas seis semanas epidemiológicas na Regional Metropolitana II }\begin{center}
% latex table generated in R 3.3.0 by xtable 1.8-2 package
% Tue Aug 23 12:43:26 2016
\begin{tabular}{c|ccccccc}
  \hline
SE & temperatura & tweet & casos notif & casos preditos & ICmin & ICmax & incidência \\ 
  \hline
201628 & 18 & 0 & 22 & 23 & 22 & 23 & 1 \\ 
  201629 & 16 & 0 & 43 & 48 & 44 & 48 & 2 \\ 
  201630 & 17 & 2 & 8 & 9 & 8 & 9 & 0 \\ 
  201631 & 18 & 2 & 4 & 4 & 4 & 4 & 0 \\ 
  201632 & 18 & 4 & 18 & 27 & 22 & 30 & 1 \\ 
  201633 & 19 & 1 & 4 & 7 & 5 & 9 & 0 \\ 
   \hline
\end{tabular}

\end{center}
\vspace{0.5cm}\begin{center}
\captionof{figure}{Casos notificados de dengue e Índice de menção em midia social sobre dengue na Regional Metropolitana II }\includegraphics[width=1.3\textwidth]{RJ/figs/tweetRJ_MetropolitanaII.png}\end{center}
\end{minipage}\hfill\begin{minipage}[t]{.30\linewidth}
\begin{mdframed}[style=sidebar,frametitle={}]
\textbf{Municipios}\begin{itemize}\ysquare Itaboraí 
\ysquare Maricá 
\ysquare Niterói 
\ysquare Rio Bonito 
\ysquare São Gonçalo 
\wsquare Silva Jardim 
\ysquare Tanguá 
\end{itemize}\BackToContents\end{mdframed}\hfill\end{minipage}\newpage\begin{minipage}[t]{.66\linewidth}
\hypertarget{LtrS}{\heading{Regional Litoral Sul}}
\includegraphics[width=0.8\textwidth]{RJ/figs/MapaRJ_LitoralSul.png}\vspace{0.5cm}\captionof{table}{Resumo das últimas seis semanas epidemiológicas na Regional Litoral Sul }\begin{center}
% latex table generated in R 3.3.1 by xtable 1.8-2 package
% Tue Sep 20 20:49:24 2016
\begin{tabular}{c|ccccccc}
  \hline
SE & temperatura & tweet & casos notif & casos preditos & ICmin & ICmax & incidência \\ 
  \hline
201632 & 16 & 2 & 12 & 12 & 12 & 12 & 5 \\ 
  201633 & 17 & 0 & 11 & 11 & 11 & 11 & 4 \\ 
  201634 & 15 & 0 & 17 & 17 & 17 & 17 & 6 \\ 
  201635 & 18 &  & 18 & 18 & 18 & 18 & 7 \\ 
  201636 & 19 &  & 13 & 13 & 13 & 14 & 5 \\ 
  201637 & 20 &  & 11 & 12 & 11 & 12 & 4 \\ 
   \hline
\end{tabular}

\end{center}
\vspace{0.5cm}\begin{center}
\captionof{figure}{Casos notificados de dengue e Índice de menção em midia social sobre dengue na Regional Litoral Sul }\includegraphics[width=1.3\textwidth]{RJ/figs/tweetRJ_LitoralSul.png}\end{center}
\end{minipage}\hfill\begin{minipage}[t]{.30\linewidth}
\begin{mdframed}[style=sidebar,frametitle={}]
\textbf{Municipios}\begin{itemize}\gsquare Angra dos Reis 
\wsquare Mangaratiba 
\gsquare Paraty 
\end{itemize}\BackToContents\end{mdframed}\hfill\end{minipage}\newpage\begin{minipage}[t]{.66\linewidth}
\hypertarget{MdPr}{\heading{Regional Médio Paraíba}}
\includegraphics[width=0.8\textwidth]{RJ/figs/MapaRJ_MedioParaiba.png}\vspace{0.5cm}\captionof{table}{Resumo das últimas seis semanas epidemiológicas na Regional Médio Paraíba }\begin{center}
% latex table generated in R 3.3.1 by xtable 1.8-2 package
% Fri Sep  9 11:31:16 2016
\begin{tabular}{c|ccccccc}
  \hline
SE & temperatura & tweet & casos notif & casos preditos & ICmin & ICmax & incidência \\ 
  \hline
201630 & 15 & 3 & 9 & 9 & 9 & 9 & 1 \\ 
  201631 & 16 & 5 & 9 & 9 & 9 & 9 & 1 \\ 
  201632 & 15 & 4 & 39 & 41 & 39 & 42 & 4 \\ 
  201633 & 16 & 1 & 7 & 7 & 7 & 7 & 1 \\ 
  201634 & 15 & 1 & 41 & 54 & 48 & 57 & 5 \\ 
  201635 & 18 &  & 59 & 118 & 109 & 122 & 7 \\ 
   \hline
\end{tabular}

\end{center}
\vspace{0.5cm}\begin{center}
\captionof{figure}{Casos notificados de dengue e Índice de menção em midia social sobre dengue na Regional Médio Paraíba }\includegraphics[width=1.3\textwidth]{RJ/figs/tweetRJ_MedioParaiba.png}\end{center}
\end{minipage}\hfill\begin{minipage}[t]{.30\linewidth}
\begin{mdframed}[style=sidebar,frametitle={}]
\textbf{Municipios}\begin{itemize}\wsquare Itatiaia 
\rgsquare Resende 
\wsquare Porto Real 
\wsquare Quatis 
\osquare Barra Mansa 
\ysquare Volta Redonda 
\wsquare Rio Claro 
\wsquare Rio das Flores 
\ysquare Valença 
\wsquare Barra do Piraí 
\gsquare Pinheiral 
\rgsquare Piraí 
\end{itemize}\BackToContents\end{mdframed}\hfill\end{minipage}\newpage\begin{minipage}[t]{.66\linewidth}
\hypertarget{CntS}{\heading{Regional Centro Sul}}
\includegraphics[width=0.8\textwidth]{RJ/figs/MapaRJ_CentroSul.png}\vspace{0.5cm}\captionof{table}{Resumo das últimas seis semanas epidemiológicas na Regional Centro Sul }\begin{center}
% latex table generated in R 3.3.0 by xtable 1.8-2 package
% Tue Aug 23 12:43:32 2016
\begin{tabular}{c|ccccccc}
  \hline
SE & temperatura & tweet & casos notif & casos preditos & ICmin & ICmax & incidência \\ 
  \hline
201628 & 18 & 0 & 1 & 1 & 1 & 1 & 0 \\ 
  201629 & 16 & 0 & 1 & 1 & 1 & 1 & 0 \\ 
  201630 & 17 & 0 & 36 & 38 & 37 & 40 & 11 \\ 
  201631 & 18 & 0 & 0 & 0 & 0 & 0 & 0 \\ 
  201632 & 18 & 0 & 1 & 1 & 1 & 1 & 0 \\ 
  201633 & 19 & 0 & 0 & 0 & 0 & 0 & 0 \\ 
   \hline
\end{tabular}

\end{center}
\vspace{0.5cm}\begin{center}
\captionof{figure}{Casos notificados de dengue e Índice de menção em midia social sobre dengue na Regional Centro Sul }\includegraphics[width=1.3\textwidth]{RJ/figs/tweetRJ_CentroSul.png}\end{center}
\end{minipage}\hfill\begin{minipage}[t]{.30\linewidth}
\begin{mdframed}[style=sidebar,frametitle={}]
\textbf{Municipios}\begin{itemize}\wsquare Paracambi 
\wsquare Mendes 
\ysquare Engenheiro Paulo de Frontin 
\wsquare Miguel Pereira 
\ysquare Vassouras 
\wsquare Paty do Alferes 
\wsquare Paraíba do Sul 
\ysquare Comendador Levy Gasparian 
\osquare Três Rios 
\wsquare Areal 
\wsquare Sapucaia 
\end{itemize}\BackToContents\end{mdframed}\hfill\end{minipage}\newpage\begin{minipage}[t]{.66\linewidth}
\hypertarget{Srrn}{\heading{Regional Serrana}}
\includegraphics[width=0.8\textwidth]{RJ/figs/MapaRJ_Serrana.png}\vspace{0.5cm}\captionof{table}{Resumo das últimas seis semanas epidemiológicas na Regional Serrana }\begin{center}
% latex table generated in R 3.3.1 by xtable 1.8-2 package
% Tue Oct 25 11:50:10 2016
\begin{tabular}{c|ccccccc}
  \hline
SE & temperatura & tweet & casos notif & casos preditos & ICmin & ICmax & incidência \\ 
  \hline
201637 & 20 & 0 & 50 & 51 & 50 & 52 & 5 \\ 
  201638 & 19 & 1 & 0 & 0 & 0 & 0 & 0 \\ 
  201639 & 19 & 2 & 0 & 0 & 0 & 0 & 0 \\ 
  201640 & 19 & 0 & 0 & 0 & 0 & 0 & 0 \\ 
  201641 & 21 & 2 & 0 & 0 & 0 & 0 & 0 \\ 
  201642 & 23 &  & 0 & 0 & 0 & 0 & 0 \\ 
   \hline
\end{tabular}

\end{center}
\vspace{0.5cm}\begin{center}
\captionof{figure}{Casos notificados de dengue e Índice de menção em midia social sobre dengue na Regional Serrana }\includegraphics[width=1.3\textwidth]{RJ/figs/tweetRJ_Serrana.png}\end{center}
\end{minipage}\hfill\begin{minipage}[t]{.30\linewidth}
\begin{mdframed}[style=sidebar,frametitle={}]
\textbf{Municipios}\begin{itemize}\ysquare Petrópolis 
\wsquare Teresópolis 
\wsquare São José do Vale do Rio Preto 
\wsquare Sumidouro 
\ysquare Nova Friburgo 
\wsquare Bom Jardim 
\wsquare Cantagalo 
\ysquare Carmo 
\wsquare Duas Barras 
\wsquare Macuco 
\ysquare Cordeiro 
\wsquare Trajano de Moraes 
\wsquare Santa Maria Madalena 
\wsquare São Sebastião do Alto 
\ysquare Guapimirim 
\wsquare Cachoeiras de Macacu 
\end{itemize}\BackToContents\end{mdframed}\hfill\end{minipage}\newpage\begin{minipage}[t]{.66\linewidth}
\hypertarget{BxdL}{\heading{Regional Baixada Litorânea}}
\includegraphics[width=0.8\textwidth]{RJ/figs/MapaRJ_BaixadaLitoranea.png}\vspace{0.5cm}\captionof{table}{Resumo das últimas seis semanas epidemiológicas na Regional Baixada Litorânea }\begin{center}
% latex table generated in R 3.3.1 by xtable 1.8-2 package
% Tue Sep  6 10:49:28 2016
\begin{tabular}{c|ccccccc}
  \hline
SE & temperatura & tweet & casos notif & casos preditos & ICmin & ICmax & incidência \\ 
  \hline
201630 & 20 & 0 & 1 & 1 & 1 & 1 & 0 \\ 
  201631 & 20 & 0 & 3 & 3 & 3 & 3 & 0 \\ 
  201632 & 21 & 0 & 4 & 4 & 4 & 4 & 1 \\ 
  201633 & 22 & 0 & 1 & 1 & 1 & 1 & 0 \\ 
  201634 & 19 & 0 & 0 & 0 & 0 & 0 & 0 \\ 
  201635 &  &  & 1 & 1 & 1 & 1 & 0 \\ 
   \hline
\end{tabular}

\end{center}
\vspace{0.5cm}\begin{center}
\captionof{figure}{Casos notificados de dengue e Índice de menção em midia social sobre dengue na Regional Baixada Litorânea }\includegraphics[width=1.3\textwidth]{RJ/figs/tweetRJ_BaixadaLitoranea.png}\end{center}
\end{minipage}\hfill\begin{minipage}[t]{.30\linewidth}
\begin{mdframed}[style=sidebar,frametitle={}]
\textbf{Municipios}\begin{itemize}\ysquare Araruama 
\wsquare Armação dos Búzios 
\ysquare Arraial do Cabo 
\wsquare Cabo Frio 
\ysquare Casimiro de Abreu 
\wsquare Iguaba Grande 
\ysquare Rio das Ostras 
\wsquare São Pedro da Aldeia 
\wsquare Saquarema 
\end{itemize}\BackToContents\end{mdframed}\hfill\end{minipage}\newpage\begin{minipage}[t]{.66\linewidth}
\hypertarget{Nort}{\heading{Regional Norte}}
\includegraphics[width=0.8\textwidth]{RJ/figs/MapaRJ_Norte.png}\vspace{0.5cm}\captionof{table}{Resumo das últimas seis semanas epidemiológicas na Regional Norte }\begin{center}
% latex table generated in R 3.3.1 by xtable 1.8-2 package
% Tue Sep 13 09:55:02 2016
\begin{tabular}{c|ccccccc}
  \hline
SE & temperatura & tweet & casos notif & casos preditos & ICmin & ICmax & incidência \\ 
  \hline
201631 & 18 & 3 & 23 & 24 & 23 & 24 & 3 \\ 
  201632 & 19 & 3 & 17 & 18 & 17 & 18 & 2 \\ 
  201633 & 20 & 0 & 27 & 32 & 30 & 33 & 3 \\ 
  201634 & 17 & 0 & 6 & 6 & 6 & 6 & 1 \\ 
  201635 & 20 &  & 2 & 2 & 2 & 2 & 0 \\ 
  201636 &  &  & 6 & 12 & 9 & 15 & 1 \\ 
   \hline
\end{tabular}

\end{center}
\vspace{0.5cm}\begin{center}
\captionof{figure}{Casos notificados de dengue e Índice de menção em midia social sobre dengue na Regional Norte }\includegraphics[width=1.3\textwidth]{RJ/figs/tweetRJ_Norte.png}\end{center}
\end{minipage}\hfill\begin{minipage}[t]{.30\linewidth}
\begin{mdframed}[style=sidebar,frametitle={}]
\textbf{Municipios}\begin{itemize}\osquare Macaé 
\wsquare Conceição de Macabu 
\wsquare Carapebus 
\ysquare Quissamã 
\wsquare São Fidélis 
\ysquare Campos dos Goytacazes 
\wsquare São João da Barra 
\wsquare São Francisco de Itabapoana 
\end{itemize}\BackToContents\end{mdframed}\hfill\end{minipage}\newpage\begin{minipage}[t]{.66\linewidth}
\hypertarget{Nrst}{\heading{Regional Noroeste}}
\includegraphics[width=0.8\textwidth]{RJ/figs/MapaRJ_Noroeste.png}\vspace{0.5cm}\captionof{table}{Resumo das últimas seis semanas epidemiológicas na Regional Noroeste }\begin{center}
% latex table generated in R 3.3.1 by xtable 1.8-2 package
% Tue Oct 18 10:05:44 2016
\begin{tabular}{c|ccccccc}
  \hline
SE & temperatura & tweet & casos notif & casos preditos & ICmin & ICmax & incidência \\ 
  \hline
201636 & 22 & 1 & 1 & 1 & 1 & 1 & 0 \\ 
  201637 & 22 & 0 & 2 & 2 & 2 & 2 & 1 \\ 
  201638 & 21 & 0 & 0 & 0 & 0 & 0 & 0 \\ 
  201639 & 21 & 0 & 3 & 3 & 3 & 3 & 1 \\ 
  201640 &  & 0 & 4 & 4 & 4 & 5 & 1 \\ 
  201641 &  &  & 0 & 0 & 0 & 0 & 0 \\ 
   \hline
\end{tabular}

\end{center}
\vspace{0.5cm}\begin{center}
\captionof{figure}{Casos notificados de dengue e Índice de menção em midia social sobre dengue na Regional Noroeste }\includegraphics[width=1.3\textwidth]{RJ/figs/tweetRJ_Noroeste.png}\end{center}
\end{minipage}\hfill\begin{minipage}[t]{.30\linewidth}
\begin{mdframed}[style=sidebar,frametitle={}]
\textbf{Municipios}\begin{itemize}\wsquare Aperibé 
\wsquare Bom Jesus do Itabapoana 
\wsquare Cambuci 
\wsquare Italva 
\wsquare Itaocara 
\ysquare Itaperuna 
\wsquare Laje do Muriaé 
\wsquare Miracema 
\ysquare Natividade 
\rgsquare Porciúncula 
\wsquare Santo Antônio de Pádua 
\wsquare São José de Ubá 
\wsquare Varre-Sai 
\wsquare Cardoso Moreira 
\end{itemize}\BackToContents\end{mdframed}\hfill\end{minipage}\newpage

%-----------------------------------------------------------------------------------%	MAIN BODY - THIRD PAGE
%-----------------------------------------------------------------------------------
 %\begin{minipage}[t]{1\linewidth} % Mini page taking up 100% of the actual page

 
      \hypertarget{municips}{\heading{Resumo da situação epidemiológica da dengue nos municípios de Rio de Janeiro, na semana 13 de 2016}{6pt}}

      
     \begin{center}
            % latex table generated in R 3.3.1 by xtable 1.8-2 package
% Tue Oct 11 13:17:57 2016
\begin{longtable}{l|lllllll}
  \hline
Municipio & Regional & Temperatura & Tweets & Casos & Incidencia & Rt & Nivel \\ 
  \hline
\endhead
\hline
{\footnotesize Continua na próxima página}
\endfoot
\endlastfoot
Angra dos Reis & Litoral Sul &  & 1 & 0 & 0.0 & 0.0 & verde \\ 
  Aperibé & Noroeste &  & 0 & 0 & 0.0 & 0.0 & verde \\ 
  Araruama & Baixada Litorânea &  & 0 & 0 & 0.0 & 0.0 & verde \\ 
  Areal & Centro Sul & 18.1 & 0 & 1 & 8.4 & 0.2 & verde \\ 
  Armação dos Búzios & Baixada Litorânea &  & 0 & 0 & 0.0 & 0.0 & verde \\ 
  Arraial do Cabo & Baixada Litorânea &  & 0 & 0 & 0.0 & 0.0 & verde \\ 
  Barra do Piraí & Médio Paraíba &  & 0 & 0 & 0.0 & 0.0 & verde \\ 
  Barra Mansa & Médio Paraíba &  & 0 & 1 & 0.6 & 12.1 & verde \\ 
  Belford Roxo & Metropolitana I & 18.1 & 0 & 0 & 0.0 & 0.0 & verde \\ 
  Bom Jardim & Serrana & 18.1 & 0 & 0 & 0.0 & 0.0 & verde \\ 
  Bom Jesus do Itabapoana & Noroeste &  & 0 & 0 & 0.0 & 0.0 & verde \\ 
  Cabo Frio & Baixada Litorânea &  & 0 & 0 & 0.0 & 0.0 & verde \\ 
  Cachoeiras de Macacu & Serrana & 18.1 & 0 & 0 & 0.0 & 0.0 & verde \\ 
  Cambuci & Noroeste &  & 0 & 0 & 0.0 & 0.0 & verde \\ 
  Campos dos Goytacazes & Norte &  & 0 & 0 & 0.0 & 0.0 & verde \\ 
  Cantagalo & Serrana & 18.1 & 0 & 0 & 0.0 & 0.0 & verde \\ 
  Carapebus & Norte &  & 0 & 0 & 0.0 & 0.0 & verde \\ 
  Cardoso Moreira & Noroeste &  & 0 & 0 & 0.0 & 0.0 & verde \\ 
  Carmo & Serrana & 18.1 & 0 & 0 & 0.0 & 0.0 & verde \\ 
  Casimiro de Abreu & Baixada Litorânea &  & 0 & 0 & 0.0 & 0.0 & verde \\ 
  Comendador Levy Gasparian & Centro Sul & 18.1 & 0 & 0 & 0.0 & 0.0 & verde \\ 
  Conceição de Macabu & Norte &  & 0 & 0 & 0.0 & 0.0 & verde \\ 
  Cordeiro & Serrana & 18.1 & 0 & 0 & 0.0 & 0.0 & verde \\ 
  Duas Barras & Serrana & 18.1 & 0 & 0 & 0.0 & 0.0 & verde \\ 
  Duque de Caxias & Metropolitana I & 18.1 & 0 & 0 & 0.0 & 0.0 & verde \\ 
  Engenheiro Paulo de Frontin & Centro Sul & 18.1 & 0 & 0 & 0.0 & 0.0 & verde \\ 
  Guapimirim & Serrana & 19.6 & 0 & 0 & 0.0 & 0.0 & verde \\ 
  Iguaba Grande & Baixada Litorânea &  & 0 & 0 & 0.0 & 0.0 & verde \\ 
  Itaboraí & Metropolitana II & 19.6 & 0 & 0 & 0.0 & 0.0 & verde \\ 
  Itaguaí & Metropolitana I & 18.1 & 0 & 0 & 0.0 & 0.0 & verde \\ 
  Italva & Noroeste &  & 0 & 1 & 6.9 & 1.2 & verde \\ 
  Itaocara & Noroeste &  & 0 & 0 & 0.0 & 0.0 & verde \\ 
  Itaperuna & Noroeste &  & 0 & 3 & 4.1 & 12.1 & verde \\ 
  Itatiaia & Médio Paraíba &  & 0 & 0 & 0.0 & 0.0 & verde \\ 
  Japeri & Metropolitana I & 18.1 & 0 & 0 & 0.0 & 0.0 & verde \\ 
  Laje do Muriaé & Noroeste &  & 0 & 0 & 0.0 & 0.0 & verde \\ 
  Macaé & Norte &  & 1 & 6 & 6.1 & 2.7 & verde \\ 
  Macuco & Serrana & 18.1 & 0 & 0 & 0.0 & 0.0 & verde \\ 
  Magé & Metropolitana I & 18.1 & 0 & 1 & 0.4 & 7.1 & verde \\ 
  Mangaratiba & Litoral Sul &  & 0 & 0 & 0.0 & 0.0 & verde \\ 
  Maricá & Metropolitana II & 19.6 & 0 & 0 & 0.0 & 0.0 & verde \\ 
  Mendes & Centro Sul & 18.1 & 0 & 0 & 0.0 & 0.0 & verde \\ 
  Mesquita & Metropolitana I & 18.1 & 0 & 0 & 0.0 & 0.0 & verde \\ 
  Miguel Pereira & Centro Sul & 18.1 & 0 & 0 & 0.0 & 0.0 & verde \\ 
  Miracema & Noroeste &  & 0 & 0 & 0.0 & 0.0 & verde \\ 
  Natividade & Noroeste &  & 0 & 0 & 0.0 & 0.0 & verde \\ 
  Nilópolis & Metropolitana I & 18.1 & 0 & 0 & 0.0 & 0.0 & verde \\ 
  Niterói & Metropolitana II & 19.6 & 0 & 0 & 0.0 & 0.0 & verde \\ 
  Nova Friburgo & Serrana & 18.1 & 0 & 0 & 0.0 & 0.0 & verde \\ 
  Nova Iguaçu & Metropolitana I & 18.1 & 1 & 0 & 0.0 & 0.0 & verde \\ 
  Paracambi & Centro Sul & 18.1 & 0 & 0 & 0.0 & 0.0 & verde \\ 
  Paraíba do Sul & Centro Sul & 18.1 & 0 & 0 & 0.0 & 0.0 & verde \\ 
  Paraty & Litoral Sul &  & 0 & 0 & 0.0 & 0.0 & verde \\ 
  Paty do Alferes & Centro Sul & 18.1 & 0 & 0 & 0.0 & 0.0 & verde \\ 
  Petrópolis & Serrana & 18.1 & 0 & 0 & 0.0 & 0.0 & verde \\ 
  Pinheiral & Médio Paraíba &  & 0 & 0 & 0.0 & 0.0 & verde \\ 
  Piraí & Médio Paraíba &  & 0 & 0 & 0.0 & 0.0 & verde \\ 
  Porciúncula & Noroeste &  & 0 & 0 & 0.0 & 0.0 & verde \\ 
  Porto Real & Médio Paraíba &  & 0 & 0 & 0.0 & 0.0 & verde \\ 
  Quatis & Médio Paraíba &  & 0 & 0 & 0.0 & 0.0 & verde \\ 
  Queimados & Metropolitana I & 18.1 & 0 & 0 & 0.0 & 0.0 & verde \\ 
  Quissamã & Norte &  & 0 & 0 & 0.0 & 0.0 & verde \\ 
  Resende & Médio Paraíba &  & 0 & 0 & 0.0 & 0.0 & verde \\ 
  Rio Bonito & Metropolitana II & 18.1 & 0 & 0 & 0.0 & 0.0 & verde \\ 
  Rio Claro & Médio Paraíba &  & 0 & 0 & 0.0 & 0.0 & verde \\ 
  Rio das Flores & Médio Paraíba &  & 0 & 0 & 0.0 & 0.0 & verde \\ 
  Rio das Ostras & Baixada Litorânea &  & 0 & 0 & 0.0 & 0.0 & verde \\ 
  Rio de Janeiro & Metropolitana I & 18.1 & 2 & 10 & 0.4 & 0.4 & verde \\ 
  Santa Maria Madalena & Serrana & 18.1 & 0 & 0 & 0.0 & 0.0 & verde \\ 
  Santo Antônio de Pádua & Noroeste &  & 0 & 0 & 0.0 & 0.0 & verde \\ 
  São Fidélis & Norte &  & 0 & 0 & 0.0 & 0.0 & verde \\ 
  São Francisco de Itabapoana & Norte &  & 0 & 0 & 0.0 & 0.0 & verde \\ 
  São Gonçalo & Metropolitana II & 19.6 & 0 & 0 & 0.0 & 0.0 & verde \\ 
  São João da Barra & Norte &  & 0 & 0 & 0.0 & 0.0 & verde \\ 
  São João de Meriti & Metropolitana I & 19.6 & 0 & 0 & 0.0 & 0.0 & verde \\ 
  São José de Ubá & Noroeste &  & 0 & 0 & 0.0 & 0.0 & verde \\ 
  São José do Vale do Rio Preto & Serrana & 18.1 & 0 & 0 & 0.0 & 0.0 & verde \\ 
  São Pedro da Aldeia & Baixada Litorânea &  & 0 & 0 & 0.0 & 0.0 & verde \\ 
  São Sebastião do Alto & Serrana & 18.1 & 0 & 0 & 0.0 & 0.0 & verde \\ 
  Sapucaia & Centro Sul & 18.1 & 0 & 0 & 0.0 & 0.0 & verde \\ 
  Saquarema & Baixada Litorânea &  & 0 & 3 & 3.7 & 0.9 & verde \\ 
  Seropédica & Metropolitana I & 18.1 & 0 & 0 & 0.0 & 0.0 & verde \\ 
  Silva Jardim & Metropolitana II & 18.1 & 0 & 0 & 0.0 & 0.0 & verde \\ 
  Sumidouro & Serrana & 18.1 & 0 & 0 & 0.0 & 0.0 & verde \\ 
  Tanguá & Metropolitana II & 19.6 & 0 & 0 & 0.0 & 0.0 & verde \\ 
  Teresópolis & Serrana & 18.1 & 0 & 0 & 0.0 & 0.0 & verde \\ 
  Trajano de Moraes & Serrana & 18.1 & 0 & 0 & 0.0 & 0.0 & verde \\ 
  Três Rios & Centro Sul & 18.1 & 0 & 0 & 0.0 & 0.0 & verde \\ 
  Valença & Médio Paraíba &  & 0 & 0 & 0.0 & 0.0 & verde \\ 
  Varre-Sai & Noroeste &  & 0 & 0 & 0.0 & 0.0 & verde \\ 
  Vassouras & Centro Sul & 18.1 & 0 & 0 & 0.0 & 0.0 & verde \\ 
  Volta Redonda & Médio Paraíba &  & 0 & 1 & 0.4 & 1.1 & verde \\ 
  \hline
\end{longtable}

     \end{center}


      \BackToContents % Link back to the contents of the newsletter


\newpage

%---------------------------------------------------------------------------------
%	Variáves nas Tabelas, Créditos e Contato
%---------------------------------------------------------------------------------

\begin{minipage}[t]{1\linewidth} 

\hypertarget{vartab}{\heading{Lista das variáveis apresentadas nas tabelas:}}

\begin{description}
\item[SE =] semana epidemiológica
\item [tweet =] número de tweets indicativos de casos de dengue na cidade
\item [temp =] média das temperaturas mínimas da semana
\item [casos =] casos notificados de dengue 
\item [casos.esp =] número de casos estimados após correção pelo atraso de notificação
\item [maxcasos =] número máximo de casos estimados (IC 95\%)
\item [p(Rt1) =] probabilidade do número reprodutivo ser maior que 1 (indica transmissão)
\item [inc =] incidência por 100.000 habs
\item [nivel =] nivel do alerta (1 = mais baixo, 4 = mais alto)
\item [cor =] cor do alerta (verde, amarelo, laranja, vermelho)
\end{description}

\hypertarget{notas}{\heading{Notas}}

\begin{itemize}
\item Os dados do sinan mais recentes ainda não foram totalmente digitados. Estimamos o número esperado de casos notificados considerando o tempo até os casos serem digitados.
\item Os dados de tweets são gerados pelo Observatório de Dengue (UFMG). Os tweets são processados para exclusão de informes e outros temas relacionados a dengue.
\item Algumas vezes, os casos da última semana ainda não estao disponíveis, nesse caso, usa-se uma estimação com base na tendência de variação da série.
\end{itemize}

\hypertarget{creditos}{\heading{Créditos}}

Este é um projeto desenvolvido com apoio da SVS/MS em parceria com:

\begin{itemize}
\item Programa de Computação Científica, Fundação Oswaldo Cruz, Rio de Janeiro.
\item Escola de Matemática Aplicada, Fundação Getúlio Vargas.
\item Secretarias do Estado e Município do Rio de Janeiro.
\item Observatório de Dengue da UFMG
\item Secretaria Estadual de Saúde do Paraná.
\end{itemize}

      \BackToContents % Link back to the contents of the newsletter

\vspace{1cm}

\hline
Para mais detalhes sobre o sistema de alerta InfoDengue, consultar: \url{http://info.dengue.mat.br}\\

\textbf{Contato}: \href{alerta\_dengue@fiocruz.br}{\nolinkurl{alerta\_dengue@fiocruz.br} }
\end{minipage} % fim da pagina de creditos

\end{document} 
