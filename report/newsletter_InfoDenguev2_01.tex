%%%%%%%%%%%%%%%%%%%%%%%%%%%%%%%%%%%%%%%%%
% Boletim semanal estadual do InfoDengue
% Adapted for InfoDengue by Claudia Codeco and Thais Riback (April 2016)
%
% Created by:
% Bob Kerstetter (https://www.tug.org/texshowcase/) and extensively modified by:
% Vel (vel@latextemplates.com)
% 
% Original template downloaded from:
% http://www.LaTeXTemplates.com
%
% License:
% CC BY-NC-SA 3.0 (http://creativecommons.org/licenses/by-nc-sa/3.0/)
%%%%%%%%%%%%%%%%%%%%%%%%%%%%%%%%%%%%%%%%%

\documentclass[10pt]{article} % The default font size is 10pt; 11pt and 12pt are alternatives
%%%%%%%%%%%%%%%%%%%%%%%%%%%%%%%%%%%%%%%%%
% Professional Newsletter Template
% Structural Definitions File
% Version 1.0 (09/03/14)
%
% Created by:
% Vel (vel@latextemplates.com)
% 
% This file has been downloaded from:
% http://www.LaTeXTemplates.com
%
% License:
% CC BY-NC-SA 3.0 (http://creativecommons.org/licenses/by-nc-sa/3.0/)
%
%%%%%%%%%%%%%%%%%%%%%%%%%%%%%%%%%%%%%%%%%

%----------------------------------------------------------------------------------------
%	REQUIRED PACKAGES
%----------------------------------------------------------------------------------------

\usepackage{graphicx} % Required for including images
\usepackage[utf8]{inputenc}
\usepackage{microtype} % Improved typography
\usepackage{multicol} % Used for the two-column layout of the document
\usepackage{booktabs} % Required for nice horizontal rules in tables
\usepackage{wrapfig} % Required for in-line images
\usepackage{float} % Required for forcing figures not to float with the [H] 
\usepackage{longtable}
\usepackage{fancyhdr} % footer 
%parameter
\usepackage{Sweave}

%-------------

% -------------
% footer style
\pagestyle{fancy}
\fancyhead{}
\fancyfoot{}
\renewcommand{\headrulewidth}{0.0pt}
\renewcommand{\footrulewidth}{0.4pt}

%------------------------------------------------
% Fonts

\usepackage{charter} % Use the Charter font as the main document font
\usepackage{courier} % Use the Courier font for \texttt (monospaced) only
%\usepackage[T1]{fontenc} % Use T1 font encoding
\usepackage[utf8]{inputenc}
\usepackage[brazilian]{babel}

%------------------------------------------------
% List Separation

\usepackage{enumitem} % Required to customize the list environments
\setlist{noitemsep,nolistsep} % Remove spacing before, after and within lists for a compact look

%------------------------------------------------
% Figure and Table Caption Styles

\usepackage{caption} % Required for changing caption styles
\captionsetup[table]{labelfont={bf,sf},labelsep=period,justification=justified} % Specify the table caption style
\captionsetup[figure]{labelfont={sf,bf},labelsep=period,justification=justified, font=small} % Specify the figure caption style
\setlength{\abovecaptionskip}{10pt} % Whitespace above captions

%------------------------------------------------
% Spacing Between Paragraphs

\makeatletter
\usepackage{parskip}
\setlength{\parskip}{6pt}
\newcommand{\@minipagerestore}{\setlength{\parskip}{6pt}}
\makeatother

%----------------------------------------------------------------------------------------
%	PAGE MARGINS AND SPACINGS
%----------------------------------------------------------------------------------------

\textwidth = 7 in % Text width
\textheight = 9.5 in % Text height
\oddsidemargin = -10pt % Left side margin on odd pages
\evensidemargin = -10pt % Left side margin on even pages
\topmargin = -10pt % Top margin
\headheight = 0pt % Remove the header by setting its space to 0
\headsep = 0pt % Remove the space between the header and top of the page
\parskip = 4pt % Space between paragraph
\parindent = 0.0in % Paragraph indentation
%\pagestyle{empty} % Disable page numbering

%----------------------------------------------------------------------------------------
%	COLORS
%----------------------------------------------------------------------------------------

\usepackage[dvipsnames,svgnames]{xcolor} % Required to specify custom colors
\definecolor{altncolor}{rgb}{.0,0.0,0.8} % Dark blue
\definecolor{myred}{rgb}{.8,0,0} % Dark red
\definecolor{myorange}{rgb}{.93,.57,.0} % Dark orange
\definecolor{myyellow}{rgb}{.93,.93,.0} % yellow
\definecolor{mygreen}{rgb}{.0,.55,.27} % dark green
\definecolor{mywhite}{rgb}{1,1,1} % dark green

\usepackage[colorlinks=true, linkcolor=altncolor, anchorcolor=altncolor, citecolor=altncolor, filecolor=altncolor, menucolor=altncolor, urlcolor=altncolor]{hyperref} % Use the color defined above for all links

% --------------------------------------
% Bullet styles
\usepackage{pifont}
\newcommand{\gsquare}{\item[\color{mygreen}\ding{110}]} 
\newcommand{\ysquare}{\item[\color{myyellow}\ding{110}]} 
\newcommand{\osquare}{\item[\color{myorange}\ding{110}]} 
\newcommand{\rsquare}{\item[\color{myred}\ding{110}]} 
\newcommand{\wsquare}{\item[\color{mywhite}\ding{110}]} 


%----------------------------------------------------------------------------------------
%	BOX STYLES
%----------------------------------------------------------------------------------------

\usepackage[framemethod=TikZ]{mdframed}% Required for creating boxes
\mdfdefinestyle{sidebar}{
    linecolor=black, % Outer line color
    outerlinewidth=0.5pt, % Outer line width
    roundcorner=0pt, % Amount of corner rounding
    innertopmargin=10pt, % Top margin
    innerbottommargin=10pt, % Bottom margin
    innerrightmargin=10pt, % Right margin
    innerleftmargin=10pt, % Left margin
    backgroundcolor=white, % Box background color
    frametitlebackgroundcolor=white, % Title background color
    frametitlerule=false, % Title rule - true or false
    frametitlerulecolor=white, % Title rule color
    frametitlerulewidth=0.5pt, % Title rule width
    frametitlefont=\Large, % Title heading font specification
    font=\small
}

\mdfdefinestyle{intextbox}{
    linecolor=blue, % Outer line color
    outerlinewidth=0.5pt, % Outer line width
    roundcorner=10pt, % Amount of corner rounding
    innertopmargin=7pt, % Top margin
    innerbottommargin=7pt, % Bottom margin
    innerrightmargin=7pt, % Right margin
    innerleftmargin=7pt, % Left margin
    backgroundcolor=white, % Box background color
    frametitlebackgroundcolor=white, % Title background color
    frametitlerule=false, % Title rule - true or false
    frametitlerulecolor=white, % Title rule color
    frametitlerulewidth=0.5pt, % Title rule width
    frametitlefont=\Large % Title heading font specification
}

%----------------------------------------------------------------------------------------
%	HEADING STYLE
%----------------------------------------------------------------------------------------

\newcommand{\heading}[2]{ % Define the \heading command
%\vspace{#2} % White space above the heading
{\begin{center}\Large\textbf{#1}\end{center}} % The heading style
%\vspace{#2} % White space below the heading
}

%\newcommand{\Inicio}{\hyperlink{contents}{{\small Inicio}}} % Define a command for linking back to the contents of the newsletter
\newcommand{\BackToContents}{\hyperlink{contents}{{\small Início}}} % Include the document which specifies all packages and structural customizations for this template

\usepackage{Sweave}
\begin{document}
\Sconcordance{concordance:newsletter_InfoDenguev2_01.tex:newsletter_InfoDenguev2_01.Rnw:%
1 18 1 1 0 17 1 1 8 18 1 1 6 6 0 1 3 61 1 1 60 175 0 1 2 110 1}


%---------------------------------------------------------------------------------
%	HEADER IMAGE
%---------------------------------------------------------------------------------

\begin{figure}[H]
\centering\includegraphics[width=1\linewidth]{InfoDengue2.png}  
\end{figure}

\centerline {\color{blue}\rule{1\linewidth}{2.75pt}} % Horizontal line

%---------------------------------------------------------------------------------
%	CAIXA LATERAL - PRIMEIRA PAGINA
%--------------------------------------------------------------------------------

% Carregando dados para a primeira pagina


\begin{minipage}[t]{.30\linewidth} % Mini page taking up 30% of the actual page
\begin{mdframed}[style=sidebar,frametitle={}] % Sidebar box

%-----------------------------------------------------------

\hypertarget{contents}{\textbf{{\large Boletim Semanal}}} 

\textbf{Semana 13 de 2016} % se, ano, em pp.RData

\begin{itemize}
\item \hyperlink{estado}{O Estado} 
\end{itemize}


\textbf{Regionais de Saúde:}


\begin{itemize}\item \hyperlink{MetI}{Metropolitana I}
\item \hyperlink{MetII}{Metropolitana II}
\item \hyperlink{MedPar}{Médio Paraíba}
\item \hyperlink{CSul}{Centro Sul}
\item \hyperlink{Serra}{Serrana}
\item \hyperlink{BaixLit}{Baixada Litorânea}
\item \hyperlink{Norte}{Norte}
\item \hyperlink{Nord}{Nordeste}\end{itemize}
\centerline {\rule{.75\linewidth}{.25pt}} % Horizontal line

%-----------------------------------------------------------

\hyperlink{vartab}{Variáves nas Tabelas}

\hyperlink{notas}{Notas} % These link to their appropriate sections in the newsletter

\hyperlink{creditos}{Creditos} % These link to their appropriate sections in the newsletter

\centerline {\rule{.75\linewidth}{.25pt}} % Horizontal line

\textbf{Contato}
\begin{description}
\item alerta\_dengue@fiocruz.br  
\end{description}

\end{mdframed}
\end{minipage}\hfill % End the sidebar mini page 
\begin{minipage}[t]{.66\linewidth} % Nao pode colocar espaco acima senao ela nao fica lado a lado

%--------------------------------------------------------------------------------
%	TEXTO PRINCIPAL - PRIMEIRA PAGINA - ALERTA A NIVEL DO ESTADO
%-------------------------------------------------------------------------------
\hypertarget{titulo}{\heading{Situação da Dengue no Estado do Rio de Janeiro}{6pt}} % \hypertarget provides a label to reference using 

Desde o início do ano, 5420 casos foram registrados no estado, sendo 350 na última semana. A figura abaixo mostra as condições de transmissão em cada município.

% Mapa estadual
\includegraphics[width=0.8\textwidth]{RJ/figs/Mapa_ERJ.png}
%-----------------------------------------------------------

Dos 92 municipios, 10 encontram-se em nivel verde, 20 em nivel amarelo, 30 em nivel laranja e 1 em nivel vermelho referentes a semana epidemiológica 13-2016. Para informações mais atualizadas
sobre o município do Rio de Janeiro ter acesso ao mapa interativo do estado consultar em \href{http://info.dengue.mat.br}{\textit{Info Dengue}}.


%--------------------------------------------------------------------------------
%	PRIMEIRA PAGINA : BOX - CODIGO DE CORES
%--------------------------------------------------------------------------------
\vspace{1cm}
\begin{mdframed}[style=intextbox,frametitle={}] % Sidebar box

\hypertarget{descriptivebox}{\heading{O código de Cores}{1pt}} % \hypertarget provides a label to reference using \hyperlink{label}{link text}
As cores indicam niveis de atenção
\begin{description}
\item[Verde:] temperaturas amenas, baixa incidência de casos.      
\item[Amarelo:] temperatura propícia para a população do vetor e transmissão da dengue.
\item[Laranja:] transmissão aumentada e sustentada de dengue. 
\item[Vermelho:] incidência alta de dengue, acima dos 90\% históricos.
\end{description}
\end{mdframed}

\textbf{Na semana passada:} 10 municípios em nivel amarelo, 30 em laranja e 10 em vermelho.    

\end{minipage} % End the main body - first page mini page


%----------------------------------------------------------------------------------
%	SEGUNDA PAGINA: TEXTO PRINCIPAL - LOOP PELAS REGIONAIS
%----------------------------------------------------------------------------------

\begin{minipage}[t]{.66\linewidth}
\hypertarget{MetI}{\heading{Regional Metropolitana I}}
\includegraphics[width=0.8\textwidth]{RJ/figs/mapaRJ_Metropolitana I.png}\vspace{0.5cm}\captionof{table}{Resumo das últimas seis semanas epidemiológicas.}\begin{center}
% latex table generated in R 3.2.5 by xtable 1.8-2 package
% Tue Apr 26 14:26:04 2016
\begin{tabular}{rrr}
  \hline
1 & 2 & 3 \\ 
  \hline
1.00 & 2.00 & 4.00 \\ 
  3.00 & 4.00 & 5.00 \\ 
  3.00 & 4.00 & 5.00 \\ 
   \hline
\end{tabular}
\end{center}
\end{minipage}\hfill\begin{minipage}[t]{.30\linewidth}
\begin{mdframed}[style=sidebar,frametitle={}]
\textbf{Municipios}\begin{itemize}\item \hyperlink{Rio}{Rio de Janeiro}
\item \hyperlink{Mag}{Magé}
\item \hyperlink{SGoncalo}{São Gonçalo}
\item \hyperlink{SJMeriti}{São João de Meriti}
\item \hyperlink{Niguacu}{Nova Iguaçu}
\end{itemize}\end{mdframed}\hfill\end{minipage}\newpage\begin{minipage}[t]{.66\linewidth}
\hypertarget{MetII}{\heading{Regional Metropolitana II}}
\includegraphics[width=0.8\textwidth]{RJ/figs/mapaRJ_Metropolitana II.png}\vspace{0.5cm}\captionof{table}{Resumo das últimas seis semanas epidemiológicas.}\begin{center}
% latex table generated in R 3.2.5 by xtable 1.8-2 package
% Tue Apr 26 14:26:04 2016
\begin{tabular}{rrr}
  \hline
1 & 2 & 3 \\ 
  \hline
1.00 & 2.00 & 4.00 \\ 
  3.00 & 4.00 & 5.00 \\ 
  3.00 & 4.00 & 5.00 \\ 
   \hline
\end{tabular}
\end{center}
\end{minipage}\hfill\begin{minipage}[t]{.30\linewidth}
\begin{mdframed}[style=sidebar,frametitle={}]
\textbf{Municipios}\begin{itemize}\item \hyperlink{Rio}{Rio de Janeiro}
\item \hyperlink{Mag}{Magé}
\item \hyperlink{SGoncalo}{São Gonçalo}
\item \hyperlink{SJMeriti}{São João de Meriti}
\item \hyperlink{Niguacu}{Nova Iguaçu}
\end{itemize}\end{mdframed}\hfill\end{minipage}\newpage\begin{minipage}[t]{.66\linewidth}
\hypertarget{MedPar}{\heading{Regional Médio Paraíba}}
\includegraphics[width=0.8\textwidth]{RJ/figs/mapaRJ_Médio Paraíba.png}\vspace{0.5cm}\captionof{table}{Resumo das últimas seis semanas epidemiológicas.}\begin{center}
% latex table generated in R 3.2.5 by xtable 1.8-2 package
% Tue Apr 26 14:26:04 2016
\begin{tabular}{rrr}
  \hline
1 & 2 & 3 \\ 
  \hline
1.00 & 2.00 & 4.00 \\ 
  3.00 & 4.00 & 5.00 \\ 
  3.00 & 4.00 & 5.00 \\ 
   \hline
\end{tabular}
\end{center}
\end{minipage}\hfill\begin{minipage}[t]{.30\linewidth}
\begin{mdframed}[style=sidebar,frametitle={}]
\textbf{Municipios}\begin{itemize}\item \hyperlink{Rio}{Rio de Janeiro}
\item \hyperlink{Mag}{Magé}
\item \hyperlink{SGoncalo}{São Gonçalo}
\item \hyperlink{SJMeriti}{São João de Meriti}
\item \hyperlink{Niguacu}{Nova Iguaçu}
\end{itemize}\end{mdframed}\hfill\end{minipage}\newpage\begin{minipage}[t]{.66\linewidth}
\hypertarget{CSul}{\heading{Regional Centro Sul}}
\includegraphics[width=0.8\textwidth]{RJ/figs/mapaRJ_Centro Sul.png}\vspace{0.5cm}\captionof{table}{Resumo das últimas seis semanas epidemiológicas.}\begin{center}
% latex table generated in R 3.2.5 by xtable 1.8-2 package
% Tue Apr 26 14:26:04 2016
\begin{tabular}{rrr}
  \hline
1 & 2 & 3 \\ 
  \hline
1.00 & 2.00 & 4.00 \\ 
  3.00 & 4.00 & 5.00 \\ 
  3.00 & 4.00 & 5.00 \\ 
   \hline
\end{tabular}
\end{center}
\end{minipage}\hfill\begin{minipage}[t]{.30\linewidth}
\begin{mdframed}[style=sidebar,frametitle={}]
\textbf{Municipios}\begin{itemize}\item \hyperlink{Rio}{Rio de Janeiro}
\item \hyperlink{Mag}{Magé}
\item \hyperlink{SGoncalo}{São Gonçalo}
\item \hyperlink{SJMeriti}{São João de Meriti}
\item \hyperlink{Niguacu}{Nova Iguaçu}
\end{itemize}\end{mdframed}\hfill\end{minipage}\newpage\begin{minipage}[t]{.66\linewidth}
\hypertarget{Serra}{\heading{Regional Serrana}}
\includegraphics[width=0.8\textwidth]{RJ/figs/mapaRJ_Serrana.png}\vspace{0.5cm}\captionof{table}{Resumo das últimas seis semanas epidemiológicas.}\begin{center}
% latex table generated in R 3.2.5 by xtable 1.8-2 package
% Tue Apr 26 14:26:04 2016
\begin{tabular}{rrr}
  \hline
1 & 2 & 3 \\ 
  \hline
1.00 & 2.00 & 4.00 \\ 
  3.00 & 4.00 & 5.00 \\ 
  3.00 & 4.00 & 5.00 \\ 
   \hline
\end{tabular}
\end{center}
\end{minipage}\hfill\begin{minipage}[t]{.30\linewidth}
\begin{mdframed}[style=sidebar,frametitle={}]
\textbf{Municipios}\begin{itemize}\item \hyperlink{Rio}{Rio de Janeiro}
\item \hyperlink{Mag}{Magé}
\item \hyperlink{SGoncalo}{São Gonçalo}
\item \hyperlink{SJMeriti}{São João de Meriti}
\item \hyperlink{Niguacu}{Nova Iguaçu}
\end{itemize}\end{mdframed}\hfill\end{minipage}\newpage\begin{minipage}[t]{.66\linewidth}
\hypertarget{BaixLit}{\heading{Regional Baixada Litorânea}}
\includegraphics[width=0.8\textwidth]{RJ/figs/mapaRJ_Baixada Litorânea.png}\vspace{0.5cm}\captionof{table}{Resumo das últimas seis semanas epidemiológicas.}\begin{center}
% latex table generated in R 3.2.5 by xtable 1.8-2 package
% Tue Apr 26 14:26:05 2016
\begin{tabular}{rrr}
  \hline
1 & 2 & 3 \\ 
  \hline
1.00 & 2.00 & 4.00 \\ 
  3.00 & 4.00 & 5.00 \\ 
  3.00 & 4.00 & 5.00 \\ 
   \hline
\end{tabular}
\end{center}
\end{minipage}\hfill\begin{minipage}[t]{.30\linewidth}
\begin{mdframed}[style=sidebar,frametitle={}]
\textbf{Municipios}\begin{itemize}\item \hyperlink{Rio}{Rio de Janeiro}
\item \hyperlink{Mag}{Magé}
\item \hyperlink{SGoncalo}{São Gonçalo}
\item \hyperlink{SJMeriti}{São João de Meriti}
\item \hyperlink{Niguacu}{Nova Iguaçu}
\end{itemize}\end{mdframed}\hfill\end{minipage}\newpage\begin{minipage}[t]{.66\linewidth}
\hypertarget{Norte}{\heading{Regional Norte}}
\includegraphics[width=0.8\textwidth]{RJ/figs/mapaRJ_Norte.png}\vspace{0.5cm}\captionof{table}{Resumo das últimas seis semanas epidemiológicas.}\begin{center}
% latex table generated in R 3.2.5 by xtable 1.8-2 package
% Tue Apr 26 14:26:05 2016
\begin{tabular}{rrr}
  \hline
1 & 2 & 3 \\ 
  \hline
1.00 & 2.00 & 4.00 \\ 
  3.00 & 4.00 & 5.00 \\ 
  3.00 & 4.00 & 5.00 \\ 
   \hline
\end{tabular}
\end{center}
\end{minipage}\hfill\begin{minipage}[t]{.30\linewidth}
\begin{mdframed}[style=sidebar,frametitle={}]
\textbf{Municipios}\begin{itemize}\item \hyperlink{Rio}{Rio de Janeiro}
\item \hyperlink{Mag}{Magé}
\item \hyperlink{SGoncalo}{São Gonçalo}
\item \hyperlink{SJMeriti}{São João de Meriti}
\item \hyperlink{Niguacu}{Nova Iguaçu}
\end{itemize}\end{mdframed}\hfill\end{minipage}\newpage\begin{minipage}[t]{.66\linewidth}
\hypertarget{Nord}{\heading{Regional Nordeste}}
\includegraphics[width=0.8\textwidth]{RJ/figs/mapaRJ_Nordeste.png}\vspace{0.5cm}\captionof{table}{Resumo das últimas seis semanas epidemiológicas.}\begin{center}
% latex table generated in R 3.2.5 by xtable 1.8-2 package
% Tue Apr 26 14:26:05 2016
\begin{tabular}{rrr}
  \hline
1 & 2 & 3 \\ 
  \hline
1.00 & 2.00 & 4.00 \\ 
  3.00 & 4.00 & 5.00 \\ 
  3.00 & 4.00 & 5.00 \\ 
   \hline
\end{tabular}
\end{center}
\end{minipage}\hfill\begin{minipage}[t]{.30\linewidth}
\begin{mdframed}[style=sidebar,frametitle={}]
\textbf{Municipios}\begin{itemize}\item \hyperlink{Rio}{Rio de Janeiro}
\item \hyperlink{Mag}{Magé}
\item \hyperlink{SGoncalo}{São Gonçalo}
\item \hyperlink{SJMeriti}{São João de Meriti}
\item \hyperlink{Niguacu}{Nova Iguaçu}
\end{itemize}\end{mdframed}\hfill\end{minipage}\newpage

%-----------------------------------------------------------------------------------%	MAIN BODY - THIRD PAGE
%-----------------------------------------------------------------------------------
% \begin{minipage}[t]{1\linewidth} % Mini page taking up 100% of the actual page
% 
% %\hypertarget{reg1}{\heading{Regional Metropolitana 1: cidades}{6pt}} % \hypertarget provides a label to reference using \hyperlink{label}{link text}
% 
% 
% \subsection*{Saquarema}
% 
% \begin{itemize}
% \item Nivel atual: Amarelo
% \item Probabilidade de transmissão sustentada na próxima semana: 20\%.
% \end{itemize}
% 
% \begin{center}
% %\begin{wrapfigure}[7]{l}[0pt]{0pt} % In-line figure with text wrapping around it
% \includegraphics[width=1\textwidth]{cores_Saquarema.png}
% %\end{wrapfigure}
% 
% 
% \begin{tabular}{rrrrrrr}
% \hline
% SE & casos & casos estimados & temp & tweet & Rt & nivel \\ 
% \hline
% 201610.00 & 1074.00 & 1353.00 &   0 &   9 &   3 &   0 \\ 
% 201611.00 & 1448.00 & 2111.00 &   0 &  11 &   1 &   0 \\ 
% 201612.00 & 1177.00 & 2155.00 &   0 &  10 &   2 &   0 \\ 
% 201613.00 & 1566.00 & 4689.00 &   0 &   9 &   3 &   0 \\ 
% \hline
% \end{tabular}
% \end{center}
% 
% \subsection*{Arraial do Cabo}
% 
% \begin{itemize}
% \item Nivel atual: Amarelo
% \item Probabilidade de transmissão sustentada na próxima semana: 20\%.
% \end{itemize}
% 
% \begin{center}
% %\begin{wrapfigure}[7]{l}[0pt]{0pt} % In-line figure with text wrapping around it
% \includegraphics[width=1\textwidth]{cores_Saquarema.png}
% %\end{wrapfigure}
% 
% \begin{tabular}{rrrrrrr}
% \hline
% SE & casos & casos estimados & temp & tweet & Rt & nivel \\ 
% \hline
% 201610.00 & 1074.00 & 1353.00 &   0 &   9 &   3 &   0 \\ 
% 201611.00 & 1448.00 & 2111.00 &   0 &  11 &   1 &   0 \\ 
% 201612.00 & 1177.00 & 2155.00 &   0 &  10 &   2 &   0 \\ 
% 201613.00 & 1566.00 & 4689.00 &   0 &   9 &   3 &   0 \\ 
% \hline
% \end{tabular}
% \end{center}
% 
% \BackToContents % Link back to the contents of the newsletter
% \end{minipage}\hfill % End of the main body - second page mini page
% 

\newpage

%---------------------------------------------------------------------------------
%	Variáves nas Tabelas, Créditos e Contato
%---------------------------------------------------------------------------------

\begin{minipage}[t]{1\linewidth} 

\centering
\hypertarget{notas}{\heading{Notas}{6pt}} % \hypertarget provides a label to reference using 

\hyperlink{vartab}{\textbf{Lista das variáveis apresentadas nas tabelas:}}

\begin{description}
\item[SE =] semana epidemiológica
\item [tweet =] número de tweets indicativos de casos de dengue na cidade
\item [temp =] média das temperaturas mínimas da semana
\item [casos =] casos notificados de dengue 
\item [casos.esp =] número de casos estimados após correção pelo atraso de notificação
\item [maxcasos =] número máximo de casos estimados (IC 95\%)
\item [p(Rt1) =] probabilidade do número reprodutivo ser maior que 1 (indica transmissão)
\item [inc =] incidência por 100.000 habs
\item [nivel =] nivel do alerta (1 = mais baixo, 4 = mais alto)
\item [cor =] cor do alerta (verde, amarelo, laranja, vermelho)\\
\end{description}

\hyperlink{notas}{\textbf{Observações}}\\

\begin{itemize}
\item Os dados do sinan mais recentes ainda não foram totalmente digitados. Estimamos o número esperado de casos notificados considerando o tempo até os casos serem digitados.\\ 
\item Os dados de tweets são gerados pelo Observatório de Dengue (UFMG). Os tweets são processados para exclusão de informes e outros temas relacionados a dengue.\\
\item Algumas vezes, os casos da última semana ainda não estao disponíveis, nesse caso, usa-se uma estimação com base na tendência de variação da série.\\
\end{itemize}

\textbf{Créditos}\\
Este é um projeto desenvolvido com apoio da SVS/MS em parceria com:
\begin{itemize}
\item Programa de Computação Científica, Fundação Oswaldo Cruz, Rio de Janeiro\\
\item Escola de Matemática Aplicada, Fundação Getúlio Vargas\\
\item Secretarias do Estado e Município do Rio de Janeiro\\
\item Secretaria Estadual de Saúde do Paraná\\
\end{itemize}

Para mais detalhes sobre o sistema de alerta consultar: \href{http://info.dengue.mat.br}{\textit{Info Dengue}}\\

\textbf{Contato}: alerta\_dengue@fiocruz.br
\end{minipage} % fim da pagina de creditos

\end{document} 
