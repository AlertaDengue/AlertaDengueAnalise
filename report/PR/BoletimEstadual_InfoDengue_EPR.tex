%%%%%%%%%%%%%%%%%%%%%%%%%%%%%%%%%%%%%%%%%
% Boletim semanal estadual do InfoDengue
% Adapted for InfoDengue by Claudia Codeco and Thais Riback (May 2016)
%
% Created by:
% Bob Kerstetter (https://www.tug.org/texshowcase/) and extensively modified by:
% Vel (vel@latextemplates.com)
% 
% Original template downloaded from:
% http://www.LaTeXTemplates.com
%
% License:
% CC BY-NC-SA 3.0 (http://creativecommons.org/licenses/by-nc-sa/3.0/)
%%%%%%%%%%%%%%%%%%%%%%%%%%%%%%%%%%%%%%%%%

\documentclass[10pt]{article} % The default font size is 10pt; 11pt and 12pt are alternatives
%%%%%%%%%%%%%%%%%%%%%%%%%%%%%%%%%%%%%%%%%
% Professional Newsletter Template
% Structural Definitions File
% Version 1.0 (09/03/14)
%
% Created by:
% Vel (vel@latextemplates.com)
% 
% This file has been downloaded from:
% http://www.LaTeXTemplates.com
%
% License:
% CC BY-NC-SA 3.0 (http://creativecommons.org/licenses/by-nc-sa/3.0/)
%
%%%%%%%%%%%%%%%%%%%%%%%%%%%%%%%%%%%%%%%%%

%----------------------------------------------------------------------------------------
%	REQUIRED PACKAGES
%----------------------------------------------------------------------------------------

\usepackage{graphicx} % Required for including images
\usepackage[utf8]{inputenc}
\usepackage{microtype} % Improved typography
\usepackage{multicol} % Used for the two-column layout of the document
\usepackage{booktabs} % Required for nice horizontal rules in tables
\usepackage{wrapfig} % Required for in-line images
\usepackage{float} % Required for forcing figures not to float with the [H] 
\usepackage{longtable}
\usepackage{fancyhdr} % footer 
%parameter
\usepackage{Sweave}

%-------------

% -------------
% footer style
\pagestyle{fancy}
\fancyhead{}
\fancyfoot{}
\renewcommand{\headrulewidth}{0.0pt}
\renewcommand{\footrulewidth}{0.4pt}

%------------------------------------------------
% Fonts

\usepackage{charter} % Use the Charter font as the main document font
\usepackage{courier} % Use the Courier font for \texttt (monospaced) only
%\usepackage[T1]{fontenc} % Use T1 font encoding
\usepackage[utf8]{inputenc}
\usepackage[brazilian]{babel}

%------------------------------------------------
% List Separation

\usepackage{enumitem} % Required to customize the list environments
\setlist{noitemsep,nolistsep} % Remove spacing before, after and within lists for a compact look

%------------------------------------------------
% Figure and Table Caption Styles

\usepackage{caption} % Required for changing caption styles
\captionsetup[table]{labelfont={bf,sf},labelsep=period,justification=justified} % Specify the table caption style
\captionsetup[figure]{labelfont={sf,bf},labelsep=period,justification=justified, font=small} % Specify the figure caption style
\setlength{\abovecaptionskip}{10pt} % Whitespace above captions

%------------------------------------------------
% Spacing Between Paragraphs

\makeatletter
\usepackage{parskip}
\setlength{\parskip}{6pt}
\newcommand{\@minipagerestore}{\setlength{\parskip}{6pt}}
\makeatother

%----------------------------------------------------------------------------------------
%	PAGE MARGINS AND SPACINGS
%----------------------------------------------------------------------------------------

\textwidth = 7 in % Text width
\textheight = 9.5 in % Text height
\oddsidemargin = -10pt % Left side margin on odd pages
\evensidemargin = -10pt % Left side margin on even pages
\topmargin = -10pt % Top margin
\headheight = 0pt % Remove the header by setting its space to 0
\headsep = 0pt % Remove the space between the header and top of the page
\parskip = 4pt % Space between paragraph
\parindent = 0.0in % Paragraph indentation
%\pagestyle{empty} % Disable page numbering

%----------------------------------------------------------------------------------------
%	COLORS
%----------------------------------------------------------------------------------------

\usepackage[dvipsnames,svgnames]{xcolor} % Required to specify custom colors
\definecolor{altncolor}{rgb}{.0,0.0,0.8} % Dark blue
\definecolor{myred}{rgb}{.8,0,0} % Dark red
\definecolor{myorange}{rgb}{.93,.57,.0} % Dark orange
\definecolor{myyellow}{rgb}{.93,.93,.0} % yellow
\definecolor{mygreen}{rgb}{.0,.55,.27} % dark green
\definecolor{mywhite}{rgb}{1,1,1} % dark green

\usepackage[colorlinks=true, linkcolor=altncolor, anchorcolor=altncolor, citecolor=altncolor, filecolor=altncolor, menucolor=altncolor, urlcolor=altncolor]{hyperref} % Use the color defined above for all links

% --------------------------------------
% Bullet styles
\usepackage{pifont}
\newcommand{\gsquare}{\item[\color{mygreen}\ding{110}]} 
\newcommand{\ysquare}{\item[\color{myyellow}\ding{110}]} 
\newcommand{\osquare}{\item[\color{myorange}\ding{110}]} 
\newcommand{\rsquare}{\item[\color{myred}\ding{110}]} 
\newcommand{\wsquare}{\item[\color{mywhite}\ding{110}]} 


%----------------------------------------------------------------------------------------
%	BOX STYLES
%----------------------------------------------------------------------------------------

\usepackage[framemethod=TikZ]{mdframed}% Required for creating boxes
\mdfdefinestyle{sidebar}{
    linecolor=black, % Outer line color
    outerlinewidth=0.5pt, % Outer line width
    roundcorner=0pt, % Amount of corner rounding
    innertopmargin=10pt, % Top margin
    innerbottommargin=10pt, % Bottom margin
    innerrightmargin=10pt, % Right margin
    innerleftmargin=10pt, % Left margin
    backgroundcolor=white, % Box background color
    frametitlebackgroundcolor=white, % Title background color
    frametitlerule=false, % Title rule - true or false
    frametitlerulecolor=white, % Title rule color
    frametitlerulewidth=0.5pt, % Title rule width
    frametitlefont=\Large, % Title heading font specification
    font=\small
}

\mdfdefinestyle{intextbox}{
    linecolor=blue, % Outer line color
    outerlinewidth=0.5pt, % Outer line width
    roundcorner=10pt, % Amount of corner rounding
    innertopmargin=7pt, % Top margin
    innerbottommargin=7pt, % Bottom margin
    innerrightmargin=7pt, % Right margin
    innerleftmargin=7pt, % Left margin
    backgroundcolor=white, % Box background color
    frametitlebackgroundcolor=white, % Title background color
    frametitlerule=false, % Title rule - true or false
    frametitlerulecolor=white, % Title rule color
    frametitlerulewidth=0.5pt, % Title rule width
    frametitlefont=\Large % Title heading font specification
}

%----------------------------------------------------------------------------------------
%	HEADING STYLE
%----------------------------------------------------------------------------------------

\newcommand{\heading}[2]{ % Define the \heading command
%\vspace{#2} % White space above the heading
{\begin{center}\Large\textbf{#1}\end{center}} % The heading style
%\vspace{#2} % White space below the heading
}

%\newcommand{\Inicio}{\hyperlink{contents}{{\small Inicio}}} % Define a command for linking back to the contents of the newsletter
\newcommand{\BackToContents}{\hyperlink{contents}{{\small Início}}} % Include the document which specifies all packages and structural customizations for this template

% Carregando dados  (atualmente é manual para cada estado)

\fancyfoot[C]{Boletim Estadual - Paraná}
\fancyfoot[R]{Semana 38 de 2016}
\fancyfoot[L]{\href{http://info.dengue.mat.br}{InfoDengue}}

\usepackage{Sweave}
\begin{document}
\Sconcordance{concordance:BoletimEstadual_InfoDengue_EPR.tex:BoletimEstadual_InfoDengue_EPR.Rnw:%
1 18 1 1 7 5 1 1 0 33 1 1 6 20 0 1 4 66 1 1 64 596 0 1 2 5 1 1 4 2 1 1 %
4 63 1}


%---------------------------------------------------------------------------------
%	HEADER IMAGE
%---------------------------------------------------------------------------------

\begin{figure}[H]
\centering\includegraphics[width=1\linewidth]{../reportconfig/InfoDengue2.png}  
\end{figure}

\centerline {\color{altncolor}\rule{1\linewidth}{2.75pt}} % Horizontal line

%---------------------------------------------------------------------------------
%	CAIXA LATERAL - PRIMEIRA PAGINA
%--------------------------------------------------------------------------------

\begin{minipage}[t]{.30\linewidth} % Mini page taking up 30% of the actual page
\begin{mdframed}[style=sidebar,frametitle={}] % Sidebar box

%-----------------------------------------------------------

\hypertarget{contents}{\textbf{{\large Boletim Semanal}}} 

\textbf{Semana 38 de 2016} % se, ano, em pp.RData

\begin{itemize}
\item \hyperlink{estado}{O Estado} 
\end{itemize}


\textbf{Regionais de Saúde:}


\begin{itemize}\item \hyperlink{Umrm}{Umuarama}
\item \hyperlink{Prnv}{Paranavaí}
\item \hyperlink{FrnB}{Francisco Beltrão}
\item \hyperlink{Apcr}{Apucarana}
\item \hyperlink{Mrng}{Maringá}
\item \hyperlink{CmpM}{Campo Mourão}
\item \hyperlink{UndV}{União da Vitória}
\item \hyperlink{Grpv}{Guarapuava}
\item \hyperlink{PtBr}{Pato Branco}
\item \hyperlink{Cscv}{Cascavel}
\item \hyperlink{Cnrt}{Cianorte}
\item \hyperlink{Lndr}{Londrina}
\item \hyperlink{FzdI}{Foz do Iguaçu}
\item \hyperlink{Irat}{Iratí}
\item \hyperlink{PntG}{Ponta Grossa}
\item \hyperlink{Crtb}{Curitiba}
\item \hyperlink{Prng}{Paranaguá}
\item \hyperlink{CrnP}{Cornélio Procópio}
\item \hyperlink{Ivpr}{Ivaiporã}
\item \hyperlink{Told}{Toledo}
\item \hyperlink{Jcrz}{Jacarezinho}
\item \hyperlink{TlmB}{Telêmaco Borba}\end{itemize}
\textbf{Municípios:}

\begin{itemize}
\item \hyperlink{municips}{Tabela de Municípios} 
\end{itemize}

\centerline {\rule{.75\linewidth}{.25pt}} % Horizontal line

%-----------------------------------------------------------

\hyperlink{vartab}{Variáveis nas Tabelas}

\hyperlink{notas}{Notas} % These link to their appropriate sections in the newsletter

\hyperlink{creditos}{Creditos} % These link to their appropriate sections in the newsletter

\centerline {\rule{.75\linewidth}{.25pt}} % Horizontal line

\textbf{Contato}
\begin{description}
\item \href{mailto:alerta\_dengue@fiocruz.br}{alerta\_dengue@fiocruz.br}  
\end{description}

\end{mdframed}
\end{minipage}\hfill % End the sidebar mini page 
\begin{minipage}[t]{.66\linewidth} % Nao pode colocar espaco acima senao ela nao fica lado a lado

%--------------------------------------------------------------------------------
%	TEXTO PRINCIPAL - PRIMEIRA PAGINA - ALERTA A NIVEL DO ESTADO
%-------------------------------------------------------------------------------
\hypertarget{estado}{\heading{Situação da Dengue no Estado do Paraná}{6pt}} % \hypertarget provides a label to reference using 

Desde o início do ano, 193938 casos foram registrados no estado, sendo 346 na última semana. A figura abaixo mostra as condições de transmissão em cada município.

% Mapa estadual
\includegraphics[width=0.8\textwidth]{figs/Mapa_EPR.png}
%-----------------------------------------------------------

Dos 399 municipios, 398 encontram-se em nivel verde, 1 em nivel amarelo, 0 em nivel laranja e 0 em nivel vermelho referentes a semana epidemiológica 38-2016. Para informações mais atualizadas e acesso ao mapa interativo de outras localidads consultar em \href{http://info.dengue.mat.br}{\textit{Info Dengue}}.


%--------------------------------------------------------------------------------
%	PRIMEIRA PAGINA : BOX - CODIGO DE CORES
%--------------------------------------------------------------------------------
\vspace{1cm}
\begin{mdframed}[style=intextbox,frametitle={}] % Sidebar box

\hypertarget{descriptivebox}{\heading{O código de Cores}{1pt}} % \hypertarget provides a label to reference using \hyperlink{label}{link text}
As cores indicam niveis de atenção
\begin{description}
\item[Verde:] temperaturas amenas, baixa incidência de casos.      
\item[Amarelo:] temperatura propícia para a população do vetor e transmissão da dengue.
\item[Laranja:] transmissão aumentada e sustentada de dengue. 
\item[Vermelho:] incidência alta de dengue, acima dos 90\% históricos.
\end{description}
\end{mdframed}

 \textbf{Na semana passada:} 1 municípios em nivel amarelo, 0 em laranja e 0 em vermelho.    

\end{minipage} % End the main body - first page mini page


%----------------------------------------------------------------------------------
%	SEGUNDA PAGINA: TEXTO PRINCIPAL - LOOP PELAS REGIONAIS
%----------------------------------------------------------------------------------

\begin{minipage}[t]{.66\linewidth}
\hypertarget{Umrm}{\heading{Regional Umuarama}{6pt}}
\includegraphics[width=0.8\textwidth]{figs/MapaPR_Umuarama.png}\vspace{0.5cm}\vspace{0.5cm}\begin{center}
\captionof{figure}{Casos notificados de dengue e Índice de menção em midia social sobre dengue na Regional Umuarama }\includegraphics[width=1\textwidth]{figs/tweetPR_Umuarama.png}\end{center}
\captionof{table}{Resumo das últimas seis semanas epidemiológicas na Regional Umuarama }\begin{center}
% latex table generated in R 3.3.1 by xtable 1.8-2 package
% Fri Sep  9 12:18:37 2016
\begin{tabular}{c|ccccccc}
  \hline
SE & temperatura & tweet & casos notif & casos preditos & ICmin & ICmax & incidência \\ 
  \hline
201628 & 16 &  & 8 & 8 & 8 & 8 & 3 \\ 
  201629 & 9 & 1 & 5 & 5 & 5 & 5 & 2 \\ 
  201630 & 14 &  & 7 & 7 & 7 & 7 & 3 \\ 
  201631 & 15 &  & 2 & 2 & 2 & 2 & 1 \\ 
  201632 & 13 &  & 6 & 6 & 6 & 6 & 2 \\ 
  201633 & 16 &  & 5 & 5 & 5 & 5 & 2 \\ 
   \hline
\end{tabular}

\end{center}
\small{\hyperlink{vartab}{ver descrição das variáveis}}\end{minipage}\hfill\begin{minipage}[t]{.30\linewidth}
\begin{mdframed}[style=sidebar,frametitle={}]
\textbf{\hyperlink{municips}{Municipios}}\begin{itemize}\gsquare Altônia 
\gsquare Cafezal do Sul 
\gsquare Ivaté 
\gsquare Nova Olímpia 
\gsquare Perobal 
\gsquare Tapira 
\gsquare Alto Piquiri 
\gsquare Brasilândia do Sul 
\gsquare Cruzeiro do Oeste 
\gsquare Douradina 
\gsquare Esperança Nova 
\gsquare Francisco Alves 
\gsquare Icaraíma 
\gsquare Iporã 
\gsquare Maria Helena 
\gsquare Mariluz 
\gsquare Pérola 
\gsquare São Jorge do Patrocínio 
\gsquare Umuarama 
\gsquare Alto Paraíso 
\gsquare Xambrê 
\end{itemize}\BackToContents\end{mdframed}\hfill\end{minipage}\newpage\begin{minipage}[t]{.66\linewidth}
\hypertarget{Prnv}{\heading{Regional Paranavaí}{6pt}}
\includegraphics[width=0.8\textwidth]{figs/MapaPR_Paranavai.png}\vspace{0.5cm}\vspace{0.5cm}\begin{center}
\captionof{figure}{Casos notificados de dengue e Índice de menção em midia social sobre dengue na Regional Paranavaí }\includegraphics[width=1\textwidth]{figs/tweetPR_Paranavai.png}\end{center}
\captionof{table}{Resumo das últimas seis semanas epidemiológicas na Regional Paranavaí }\begin{center}
% latex table generated in R 3.3.1 by xtable 1.8-2 package
% Fri Oct  7 11:52:34 2016
\begin{tabular}{c|ccccccc}
  \hline
SE & temperatura & tweet & casos notif & casos preditos & ICmin & ICmax & incidência \\ 
  \hline
201634 & 12 &  & 8 & 8 & 8 & 8 & 3 \\ 
  201635 & 17 &  & 5 & 5 & 5 & 5 & 2 \\ 
  201636 & 13 &  & 4 & 4 & 4 & 4 & 1 \\ 
  201637 & 16 &  & 6 & 6 & 6 & 6 & 2 \\ 
  201638 & 15 &  & 8 & 8 & 8 & 8 & 3 \\ 
  201639 & 14 &  & 2 & 2 & 2 & 2 & 1 \\ 
   \hline
\end{tabular}

\end{center}
\small{\hyperlink{vartab}{ver descrição das variáveis}}\end{minipage}\hfill\begin{minipage}[t]{.30\linewidth}
\begin{mdframed}[style=sidebar,frametitle={}]
\textbf{\hyperlink{municips}{Municipios}}\begin{itemize}\gsquare Alto Paraná 
\gsquare Amaporã 
\gsquare Cruzeiro do Sul 
\gsquare Guairaçá 
\gsquare Inajá 
\gsquare Jardim Olinda 
\gsquare Mirador 
\gsquare Paranapoema 
\gsquare Planaltina do Paraná 
\gsquare Santa Cruz de Monte Castelo 
\gsquare Santa Isabel do Ivaí 
\gsquare Diamante do Norte 
\gsquare Itaúna do Sul 
\gsquare Loanda 
\gsquare Marilena 
\gsquare Nova Aliança do Ivaí 
\gsquare Nova Londrina 
\gsquare Paraíso do Norte 
\gsquare Paranavaí 
\gsquare Porto Rico 
\gsquare Querência do Norte 
\gsquare Santa Mônica 
\gsquare Santo Antônio do Caiuá 
\gsquare São Carlos do Ivaí 
\gsquare São João do Caiuá 
\gsquare São Pedro do Paraná 
\gsquare Tamboara 
\gsquare Terra Rica 
\end{itemize}\BackToContents\end{mdframed}\hfill\end{minipage}\newpage\begin{minipage}[t]{.66\linewidth}
\hypertarget{FrnB}{\heading{Regional Francisco Beltrão}{6pt}}
\includegraphics[width=0.8\textwidth]{figs/MapaPR_FranciscoBeltrao.png}\vspace{0.5cm}\vspace{0.5cm}\begin{center}
\captionof{figure}{Casos notificados de dengue e Índice de menção em midia social sobre dengue na Regional Francisco Beltrão }\includegraphics[width=1\textwidth]{figs/tweetPR_FranciscoBeltrao.png}\end{center}
\captionof{table}{Resumo das últimas seis semanas epidemiológicas na Regional Francisco Beltrão }\begin{center}
% latex table generated in R 3.3.1 by xtable 1.8-2 package
% Thu Sep 22 13:06:14 2016
\begin{tabular}{c|ccccccc}
  \hline
SE & temperatura & tweet & casos notif & casos preditos & ICmin & ICmax & incidência \\ 
  \hline
201632 & 11 &  & 7 & 7 & 7 & 7 & 2 \\ 
  201633 & 16 &  & 5 & 5 & 5 & 5 & 1 \\ 
  201634 & 12 &  & 10 & 10 & 10 & 10 & 3 \\ 
  201635 & 15 &  & 9 & 9 & 9 & 9 & 3 \\ 
  201636 & 10 &  & 5 & 5 & 5 & 5 & 1 \\ 
  201637 & 14 &  & 4 & 4 & 4 & 5 & 1 \\ 
   \hline
\end{tabular}

\end{center}
\small{\hyperlink{vartab}{ver descrição das variáveis}}\end{minipage}\hfill\begin{minipage}[t]{.30\linewidth}
\begin{mdframed}[style=sidebar,frametitle={}]
\textbf{\hyperlink{municips}{Municipios}}\begin{itemize}\gsquare Ampére 
\gsquare Capanema 
\gsquare Nova Prata do Iguaçu 
\gsquare Planalto 
\gsquare Realeza 
\gsquare Salgado Filho 
\gsquare Santo Antônio do Sudoeste 
\gsquare Barracão 
\gsquare Bela Vista da Caroba 
\gsquare Boa Esperança do Iguaçu 
\gsquare Bom Jesus do Sul 
\gsquare Cruzeiro do Iguaçu 
\gsquare Dois Vizinhos 
\gsquare Enéas Marques 
\gsquare Flor da Serra do Sul 
\gsquare Francisco Beltrão 
\gsquare Manfrinópolis 
\gsquare Marmeleiro 
\gsquare Nova Esperança do Sudoeste 
\gsquare Pérola d'Oeste 
\gsquare Pinhal de São Bento 
\gsquare Pranchita 
\gsquare Renascença 
\gsquare Salto do Lontra 
\gsquare Santa Izabel do Oeste 
\gsquare São Jorge d'Oeste 
\gsquare Verê 
\end{itemize}\BackToContents\end{mdframed}\hfill\end{minipage}\newpage\begin{minipage}[t]{.66\linewidth}
\hypertarget{Apcr}{\heading{Regional Apucarana}{6pt}}
\includegraphics[width=0.8\textwidth]{figs/MapaPR_Apucarana.png}\vspace{0.5cm}\vspace{0.5cm}\begin{center}
\captionof{figure}{Casos notificados de dengue e Índice de menção em midia social sobre dengue na Regional Apucarana }\includegraphics[width=1\textwidth]{figs/tweetPR_Apucarana.png}\end{center}
\captionof{table}{Resumo das últimas seis semanas epidemiológicas na Regional Apucarana }\begin{center}
% latex table generated in R 3.3.1 by xtable 1.8-2 package
% Thu Sep 15 13:58:59 2016
\begin{tabular}{c|ccccccc}
  \hline
SE & temperatura & tweet & casos notif & casos preditos & ICmin & ICmax & incidência \\ 
  \hline
201631 & 15 &  & 9 & 9 & 9 & 9 & 2 \\ 
  201632 & 13 &  & 8 & 8 & 8 & 8 & 2 \\ 
  201633 & 16 & 0 & 10 & 10 & 10 & 10 & 3 \\ 
  201634 & 12 &  & 7 & 7 & 7 & 7 & 2 \\ 
  201635 & 17 &  & 5 & 5 & 5 & 5 & 1 \\ 
  201636 & 13 &  & 1 & 1 & 1 & 1 & 0 \\ 
   \hline
\end{tabular}

\end{center}
\small{\hyperlink{vartab}{ver descrição das variáveis}}\end{minipage}\hfill\begin{minipage}[t]{.30\linewidth}
\begin{mdframed}[style=sidebar,frametitle={}]
\textbf{\hyperlink{municips}{Municipios}}\begin{itemize}\gsquare Arapongas 
\gsquare Borrazópolis 
\gsquare Cambira 
\gsquare Jandaia do Sul 
\gsquare Marilândia do Sul 
\gsquare São Pedro do Ivaí 
\gsquare Apucarana 
\gsquare Bom Sucesso 
\gsquare Califórnia 
\gsquare Faxinal 
\gsquare Grandes Rios 
\gsquare Kaloré 
\gsquare Marumbi 
\gsquare Mauá da Serra 
\gsquare Novo Itacolomi 
\gsquare Rio Bom 
\gsquare Sabáudia 
\end{itemize}\BackToContents\end{mdframed}\hfill\end{minipage}\newpage\begin{minipage}[t]{.66\linewidth}
\hypertarget{Mrng}{\heading{Regional Maringá}{6pt}}
\includegraphics[width=0.8\textwidth]{figs/MapaPR_Maringa.png}\vspace{0.5cm}\vspace{0.5cm}\begin{center}
\captionof{figure}{Casos notificados de dengue e Índice de menção em midia social sobre dengue na Regional Maringá }\includegraphics[width=1\textwidth]{figs/tweetPR_Maringa.png}\end{center}
\captionof{table}{Resumo das últimas seis semanas epidemiológicas na Regional Maringá }\begin{center}
% latex table generated in R 3.3.0 by xtable 1.8-2 package
% Tue Aug 23 12:42:07 2016
\begin{tabular}{c|ccccccc}
  \hline
SE & temperatura & tweet & casos notif & casos preditos & ICmin & ICmax & incidência \\ 
  \hline
201628 & 16 &  & 32 & 34 & 32 & 35 & 4 \\ 
  201629 & 9 & 2 & 18 & 19 & 18 & 19 & 2 \\ 
  201630 & 14 &  & 43 & 49 & 46 & 51 & 5 \\ 
  201631 & 15 &  & 33 & 40 & 37 & 41 & 4 \\ 
  201632 & 13 &  & 41 & 54 & 48 & 58 & 5 \\ 
  201633 & 16 &  & 15 & 23 & 18 & 25 & 2 \\ 
   \hline
\end{tabular}

\end{center}
\small{\hyperlink{vartab}{ver descrição das variáveis}}\end{minipage}\hfill\begin{minipage}[t]{.30\linewidth}
\begin{mdframed}[style=sidebar,frametitle={}]
\textbf{\hyperlink{municips}{Municipios}}\begin{itemize}\gsquare Atalaia 
\gsquare Doutor Camargo 
\gsquare Floraí 
\gsquare Flórida 
\gsquare Itaguajé 
\gsquare Lobato 
\gsquare Mandaguari 
\gsquare Maringá 
\gsquare Munhoz de Melo 
\gsquare Paiçandu 
\gsquare Presidente Castelo Branco 
\gsquare Santo Inácio 
\gsquare São Jorge do Ivaí 
\gsquare Sarandi 
\gsquare Uniflor 
\gsquare Ângulo 
\gsquare Astorga 
\gsquare Colorado 
\gsquare Floresta 
\gsquare Iguaraçu 
\gsquare Itambé 
\gsquare Ivatuba 
\gsquare Mandaguaçu 
\gsquare Marialva 
\gsquare Nossa Senhora das Graças 
\gsquare Nova Esperança 
\gsquare Ourizona 
\gsquare Paranacity 
\gsquare Santa Fé 
\gsquare Santa Inês 
\end{itemize}\BackToContents\end{mdframed}\hfill\end{minipage}\newpage\begin{minipage}[t]{.66\linewidth}
\hypertarget{CmpM}{\heading{Regional Campo Mourão}{6pt}}
\includegraphics[width=0.8\textwidth]{figs/MapaPR_CampoMourao.png}\vspace{0.5cm}\vspace{0.5cm}\begin{center}
\captionof{figure}{Casos notificados de dengue e Índice de menção em midia social sobre dengue na Regional Campo Mourão }\includegraphics[width=1\textwidth]{figs/tweetPR_CampoMourao.png}\end{center}
\captionof{table}{Resumo das últimas seis semanas epidemiológicas na Regional Campo Mourão }\begin{center}
% latex table generated in R 3.3.1 by xtable 1.8-2 package
% Thu Sep 22 13:06:30 2016
\begin{tabular}{c|ccccccc}
  \hline
SE & temperatura & tweet & casos notif & casos preditos & ICmin & ICmax & incidência \\ 
  \hline
201632 & 14 &  & 10 & 10 & 10 & 10 & 3 \\ 
  201633 & 17 &  & 2 & 2 & 2 & 2 & 1 \\ 
  201634 & 13 &  & 4 & 4 & 4 & 4 & 1 \\ 
  201635 & 18 &  & 2 & 2 & 2 & 2 & 1 \\ 
  201636 &  &  & 2 & 2 & 2 & 2 & 1 \\ 
  201637 &  &  & 4 & 5 & 4 & 5 & 1 \\ 
   \hline
\end{tabular}

\end{center}
\small{\hyperlink{vartab}{ver descrição das variáveis}}\end{minipage}\hfill\begin{minipage}[t]{.30\linewidth}
\begin{mdframed}[style=sidebar,frametitle={}]
\textbf{\hyperlink{municips}{Municipios}}\begin{itemize}\gsquare Barbosa Ferraz 
\gsquare Campina da Lagoa 
\gsquare Campo Mourão 
\gsquare Corumbataí do Sul 
\gsquare Engenheiro Beltrão 
\gsquare Fênix 
\gsquare Nova Cantu 
\gsquare Quinta do Sol 
\gsquare Roncador 
\gsquare Altamira do Paraná 
\gsquare Araruna 
\gsquare Boa Esperança 
\gsquare Farol 
\gsquare Goioerê 
\gsquare Iretama 
\gsquare Janiópolis 
\gsquare Juranda 
\gsquare Luiziana 
\gsquare Mamborê 
\gsquare Moreira Sales 
\gsquare Peabiru 
\gsquare Quarto Centenário 
\gsquare Rancho Alegre D'Oeste 
\gsquare Terra Boa 
\gsquare Ubiratã 
\end{itemize}\BackToContents\end{mdframed}\hfill\end{minipage}\newpage\begin{minipage}[t]{.66\linewidth}
\hypertarget{UndV}{\heading{Regional União da Vitória}{6pt}}
\includegraphics[width=0.8\textwidth]{figs/MapaPR_UniaodaVitoria.png}\vspace{0.5cm}\vspace{0.5cm}\begin{center}
\captionof{figure}{Casos notificados de dengue e Índice de menção em midia social sobre dengue na Regional União da Vitória }\includegraphics[width=1\textwidth]{figs/tweetPR_UniaodaVitoria.png}\end{center}
\captionof{table}{Resumo das últimas seis semanas epidemiológicas na Regional União da Vitória }\begin{center}
% latex table generated in R 3.3.1 by xtable 1.8-2 package
% Fri Oct  7 11:53:01 2016
\begin{tabular}{c|ccccccc}
  \hline
SE & temperatura & tweet & casos notif & casos preditos & ICmin & ICmax & incidência \\ 
  \hline
201634 & 7 &  & 0 & 0 & 0 & 0 & 0 \\ 
  201635 & 12 &  & 0 & 0 & 0 & 0 & 0 \\ 
  201636 & 10 &  & 0 & 0 & 0 & 0 & 0 \\ 
  201637 & 12 &  & 0 & 0 & 0 & 0 & 0 \\ 
  201638 & 11 &  & 0 & 0 & 0 & 0 & 0 \\ 
  201639 & 11 &  & 0 & 0 & 0 & 0 & 0 \\ 
   \hline
\end{tabular}

\end{center}
\small{\hyperlink{vartab}{ver descrição das variáveis}}\end{minipage}\hfill\begin{minipage}[t]{.30\linewidth}
\begin{mdframed}[style=sidebar,frametitle={}]
\textbf{\hyperlink{municips}{Municipios}}\begin{itemize}\gsquare Bituruna 
\gsquare Paulo Frontin 
\gsquare São Mateus do Sul 
\gsquare União da Vitória 
\gsquare Cruz Machado 
\gsquare Antônio Olinto 
\gsquare General Carneiro 
\gsquare Paula Freitas 
\gsquare Porto Vitória 
\end{itemize}\BackToContents\end{mdframed}\hfill\end{minipage}\newpage\begin{minipage}[t]{.66\linewidth}
\hypertarget{Grpv}{\heading{Regional Guarapuava}{6pt}}
\includegraphics[width=0.8\textwidth]{figs/MapaPR_Guarapuava.png}\vspace{0.5cm}\vspace{0.5cm}\begin{center}
\captionof{figure}{Casos notificados de dengue e Índice de menção em midia social sobre dengue na Regional Guarapuava }\includegraphics[width=1\textwidth]{figs/tweetPR_Guarapuava.png}\end{center}
\captionof{table}{Resumo das últimas seis semanas epidemiológicas na Regional Guarapuava }\begin{center}
% latex table generated in R 3.3.1 by xtable 1.8-2 package
% Mon Oct 31 13:04:14 2016
\begin{tabular}{c|ccccccc}
  \hline
SE & temperatura & tweet & casos notif & casos preditos & ICmin & ICmax & incidência \\ 
  \hline
201637 & 12 &  & 1 & 1 & 1 & 1 & 0 \\ 
  201638 & 11 &  & 2 & 2 & 2 & 2 & 0 \\ 
  201639 & 11 &  & 1 & 1 & 1 & 1 & 0 \\ 
  201640 & 12 &  & 0 & 0 & 0 & 0 & 0 \\ 
  201641 & 13 &  & 2 & 2 & 2 & 2 & 0 \\ 
  201642 & 16 &  & 1 & 1 & 1 & 1 & 0 \\ 
   \hline
\end{tabular}

\end{center}
\small{\hyperlink{vartab}{ver descrição das variáveis}}\end{minipage}\hfill\begin{minipage}[t]{.30\linewidth}
\begin{mdframed}[style=sidebar,frametitle={}]
\textbf{\hyperlink{municips}{Municipios}}\begin{itemize}\gsquare Boa Ventura de São Roque 
\gsquare Foz do Jordão 
\gsquare Goioxim 
\gsquare Guarapuava 
\gsquare Laranjal 
\gsquare Palmital 
\gsquare Pinhão 
\gsquare Porto Barreiro 
\gsquare Prudentópolis 
\gsquare Campina do Simão 
\gsquare Candói 
\gsquare Cantagalo 
\gsquare Laranjeiras do Sul 
\gsquare Marquinho 
\gsquare Nova Laranjeiras 
\gsquare Pitanga 
\gsquare Reserva do Iguaçu 
\gsquare Rio Bonito do Iguaçu 
\gsquare Turvo 
\gsquare Virmond 
\end{itemize}\BackToContents\end{mdframed}\hfill\end{minipage}\newpage\begin{minipage}[t]{.66\linewidth}
\hypertarget{PtBr}{\heading{Regional Pato Branco}{6pt}}
\includegraphics[width=0.8\textwidth]{figs/MapaPR_PatoBranco.png}\vspace{0.5cm}\vspace{0.5cm}\begin{center}
\captionof{figure}{Casos notificados de dengue e Índice de menção em midia social sobre dengue na Regional Pato Branco }\includegraphics[width=1\textwidth]{figs/tweetPR_PatoBranco.png}\end{center}
\captionof{table}{Resumo das últimas seis semanas epidemiológicas na Regional Pato Branco }\begin{center}
% latex table generated in R 3.3.1 by xtable 1.8-2 package
% Thu Sep 22 13:06:41 2016
\begin{tabular}{c|ccccccc}
  \hline
SE & temperatura & tweet & casos notif & casos preditos & ICmin & ICmax & incidência \\ 
  \hline
201632 & 11 & 0 & 1 & 1 & 1 & 1 & 0 \\ 
  201633 & 16 &  & 0 & 0 & 0 & 0 & 0 \\ 
  201634 & 12 &  & 1 & 1 & 1 & 1 & 0 \\ 
  201635 & 16 &  & 0 & 0 & 0 & 0 & 0 \\ 
  201636 & 12 &  & 1 & 1 & 1 & 1 & 0 \\ 
  201637 & 14 &  & 0 & 0 & 0 & 0 & 0 \\ 
   \hline
\end{tabular}

\end{center}
\small{\hyperlink{vartab}{ver descrição das variáveis}}\end{minipage}\hfill\begin{minipage}[t]{.30\linewidth}
\begin{mdframed}[style=sidebar,frametitle={}]
\textbf{\hyperlink{municips}{Municipios}}\begin{itemize}\gsquare Bom Sucesso do Sul 
\gsquare Mariópolis 
\gsquare Vitorino 
\gsquare Chopinzinho 
\gsquare Clevelândia 
\gsquare Coronel Domingos Soares 
\gsquare Coronel Vivida 
\gsquare Honório Serpa 
\gsquare Itapejara d'Oeste 
\gsquare Mangueirinha 
\gsquare Palmas 
\gsquare Pato Branco 
\gsquare São João 
\gsquare Saudade do Iguaçu 
\gsquare Sulina 
\end{itemize}\BackToContents\end{mdframed}\hfill\end{minipage}\newpage\begin{minipage}[t]{.66\linewidth}
\hypertarget{Cscv}{\heading{Regional Cascavel}{6pt}}
\includegraphics[width=0.8\textwidth]{figs/MapaPR_Cascavel.png}\vspace{0.5cm}\vspace{0.5cm}\begin{center}
\captionof{figure}{Casos notificados de dengue e Índice de menção em midia social sobre dengue na Regional Cascavel }\includegraphics[width=1\textwidth]{figs/tweetPR_Cascavel.png}\end{center}
\captionof{table}{Resumo das últimas seis semanas epidemiológicas na Regional Cascavel }\begin{center}
% latex table generated in R 3.3.0 by xtable 1.8-2 package
% Tue Aug 23 12:41:41 2016
\begin{tabular}{c|ccccccc}
  \hline
SE & temperatura & tweet & casos notif & casos preditos & ICmin & ICmax & incidência \\ 
  \hline
201628 & 16 &  & 21 & 21 & 21 & 21 & 4 \\ 
  201629 & 5 & 0 & 19 & 19 & 19 & 19 & 4 \\ 
  201630 & 13 &  & 21 & 22 & 21 & 22 & 4 \\ 
  201631 & 16 &  & 14 & 15 & 14 & 15 & 3 \\ 
  201632 & 11 &  & 15 & 17 & 15 & 17 & 3 \\ 
  201633 & 16 &  & 7 & 8 & 7 & 9 & 1 \\ 
   \hline
\end{tabular}

\end{center}
\small{\hyperlink{vartab}{ver descrição das variáveis}}\end{minipage}\hfill\begin{minipage}[t]{.30\linewidth}
\begin{mdframed}[style=sidebar,frametitle={}]
\textbf{\hyperlink{municips}{Municipios}}\begin{itemize}\gsquare Céu Azul 
\gsquare Corbélia 
\gsquare Formosa do Oeste 
\gsquare Iguatu 
\gsquare Iracema do Oeste 
\gsquare Nova Aurora 
\gsquare Santa Tereza do Oeste 
\gsquare Três Barras do Paraná 
\gsquare Boa Vista da Aparecida 
\gsquare Braganey 
\gsquare Cafelândia 
\gsquare Campo Bonito 
\gsquare Capitão Leônidas Marques 
\gsquare Cascavel 
\gsquare Catanduvas 
\gsquare Diamante do Sul 
\gsquare Espigão Alto do Iguaçu 
\gsquare Guaraniaçu 
\gsquare Ibema 
\gsquare Jesuítas 
\gsquare Lindoeste 
\gsquare Quedas do Iguaçu 
\gsquare Santa Lúcia 
\gsquare Vera Cruz do Oeste 
\gsquare Anahy 
\end{itemize}\BackToContents\end{mdframed}\hfill\end{minipage}\newpage\begin{minipage}[t]{.66\linewidth}
\hypertarget{Cnrt}{\heading{Regional Cianorte}{6pt}}
\includegraphics[width=0.8\textwidth]{figs/MapaPR_Cianorte.png}\vspace{0.5cm}\vspace{0.5cm}\begin{center}
\captionof{figure}{Casos notificados de dengue e Índice de menção em midia social sobre dengue na Regional Cianorte }\includegraphics[width=1\textwidth]{figs/tweetPR_Cianorte.png}\end{center}
\captionof{table}{Resumo das últimas seis semanas epidemiológicas na Regional Cianorte }\begin{center}
% latex table generated in R 3.3.1 by xtable 1.8-2 package
% Thu Sep 15 13:59:31 2016
\begin{tabular}{c|ccccccc}
  \hline
SE & temperatura & tweet & casos notif & casos preditos & ICmin & ICmax & incidência \\ 
  \hline
201631 & 15 & 0 & 2 & 2 & 2 & 2 & 1 \\ 
  201632 & 13 & 0 & 1 & 1 & 1 & 1 & 1 \\ 
  201633 & 16 &  & 2 & 2 & 2 & 2 & 1 \\ 
  201634 & 12 &  & 2 & 2 & 2 & 2 & 1 \\ 
  201635 & 17 &  & 0 & 0 & 0 & 0 & 0 \\ 
  201636 & 13 &  & 0 & 0 & 0 & 0 & 0 \\ 
   \hline
\end{tabular}

\end{center}
\small{\hyperlink{vartab}{ver descrição das variáveis}}\end{minipage}\hfill\begin{minipage}[t]{.30\linewidth}
\begin{mdframed}[style=sidebar,frametitle={}]
\textbf{\hyperlink{municips}{Municipios}}\begin{itemize}\gsquare Cidade Gaúcha 
\gsquare Jussara 
\gsquare Tuneiras do Oeste 
\gsquare Cianorte 
\gsquare Guaporema 
\gsquare Indianópolis 
\gsquare Japurá 
\gsquare Rondon 
\gsquare São Manoel do Paraná 
\gsquare São Tomé 
\gsquare Tapejara 
\end{itemize}\BackToContents\end{mdframed}\hfill\end{minipage}\newpage\begin{minipage}[t]{.66\linewidth}
\hypertarget{Lndr}{\heading{Regional Londrina}{6pt}}
\includegraphics[width=0.8\textwidth]{figs/MapaPR_Londrina.png}\vspace{0.5cm}\vspace{0.5cm}\begin{center}
\captionof{figure}{Casos notificados de dengue e Índice de menção em midia social sobre dengue na Regional Londrina }\includegraphics[width=1\textwidth]{figs/tweetPR_Londrina.png}\end{center}
\captionof{table}{Resumo das últimas seis semanas epidemiológicas na Regional Londrina }\begin{center}
% latex table generated in R 3.3.0 by xtable 1.8-2 package
% Tue Aug 23 12:42:17 2016
\begin{tabular}{c|ccccccc}
  \hline
SE & temperatura & tweet & casos notif & casos preditos & ICmin & ICmax & incidência \\ 
  \hline
201628 & 16 &  & 117 & 130 & 123 & 132 & 13 \\ 
  201629 & 9 & 0 & 77 & 87 & 80 & 90 & 8 \\ 
  201630 & 14 &  & 107 & 129 & 121 & 133 & 12 \\ 
  201631 & 15 &  & 101 & 126 & 114 & 132 & 11 \\ 
  201632 & 13 &  & 80 & 111 & 101 & 119 & 9 \\ 
  201633 & 16 &  & 48 & 82 & 71 & 85 & 5 \\ 
   \hline
\end{tabular}

\end{center}
\small{\hyperlink{vartab}{ver descrição das variáveis}}\end{minipage}\hfill\begin{minipage}[t]{.30\linewidth}
\begin{mdframed}[style=sidebar,frametitle={}]
\textbf{\hyperlink{municips}{Municipios}}\begin{itemize}\gsquare Ibiporã 
\gsquare Lupionópolis 
\gsquare Alvorada do Sul 
\gsquare Assaí 
\gsquare Bela Vista do Paraíso 
\gsquare Cafeara 
\gsquare Cambé 
\gsquare Centenário do Sul 
\gsquare Florestópolis 
\gsquare Guaraci 
\gsquare Jaguapitã 
\ysquare Jataizinho 
\gsquare Londrina 
\gsquare Miraselva 
\gsquare Pitangueiras 
\gsquare Porecatu 
\gsquare Prado Ferreira 
\gsquare Primeiro de Maio 
\gsquare Rolândia 
\gsquare Sertanópolis 
\gsquare Tamarana 
\end{itemize}\BackToContents\end{mdframed}\hfill\end{minipage}\newpage\begin{minipage}[t]{.66\linewidth}
\hypertarget{FzdI}{\heading{Regional Foz do Iguaçu}{6pt}}
\includegraphics[width=0.8\textwidth]{figs/MapaPR_FozdoIguacu.png}\vspace{0.5cm}\vspace{0.5cm}\begin{center}
\captionof{figure}{Casos notificados de dengue e Índice de menção em midia social sobre dengue na Regional Foz do Iguaçu }\includegraphics[width=1\textwidth]{figs/tweetPR_FozdoIguacu.png}\end{center}
\captionof{table}{Resumo das últimas seis semanas epidemiológicas na Regional Foz do Iguaçu }\begin{center}
% latex table generated in R 3.3.1 by xtable 1.8-2 package
% Fri Oct  7 11:53:22 2016
\begin{tabular}{c|ccccccc}
  \hline
SE & temperatura & tweet & casos notif & casos preditos & ICmin & ICmax & incidência \\ 
  \hline
201634 & 12 &  & 19 & 20 & 19 & 20 & 5 \\ 
  201635 & 15 &  & 12 & 12 & 12 & 12 & 3 \\ 
  201636 & 9 &  & 16 & 17 & 16 & 18 & 4 \\ 
  201637 & 14 &  & 19 & 22 & 19 & 22 & 5 \\ 
  201638 & 14 &  & 17 & 21 & 18 & 22 & 4 \\ 
  201639 & 13 &  & 16 & 22 & 18 & 24 & 4 \\ 
   \hline
\end{tabular}

\end{center}
\small{\hyperlink{vartab}{ver descrição das variáveis}}\end{minipage}\hfill\begin{minipage}[t]{.30\linewidth}
\begin{mdframed}[style=sidebar,frametitle={}]
\textbf{\hyperlink{municips}{Municipios}}\begin{itemize}\gsquare Itaipulândia 
\gsquare Matelândia 
\gsquare Foz do Iguaçu 
\gsquare Medianeira 
\gsquare Missal 
\gsquare Ramilândia 
\gsquare Santa Terezinha de Itaipu 
\gsquare São Miguel do Iguaçu 
\gsquare Serranópolis do Iguaçu 
\end{itemize}\BackToContents\end{mdframed}\hfill\end{minipage}\newpage\begin{minipage}[t]{.66\linewidth}
\hypertarget{Irat}{\heading{Regional Iratí}{6pt}}
\includegraphics[width=0.8\textwidth]{figs/MapaPR_Irati.png}\vspace{0.5cm}\vspace{0.5cm}\begin{center}
\captionof{figure}{Casos notificados de dengue e Índice de menção em midia social sobre dengue na Regional Iratí }\includegraphics[width=1\textwidth]{figs/tweetPR_Irati.png}\end{center}
\captionof{table}{Resumo das últimas seis semanas epidemiológicas na Regional Iratí }\begin{center}
% latex table generated in R 3.3.1 by xtable 1.8-2 package
% Thu Sep 22 13:06:59 2016
\begin{tabular}{c|ccccccc}
  \hline
SE & temperatura & tweet & casos notif & casos preditos & ICmin & ICmax & incidência \\ 
  \hline
201632 & 7 &  & 0 & 0 & 0 & 0 & 0 \\ 
  201633 & 11 &  & 0 & 0 & 0 & 0 & 0 \\ 
  201634 & 7 &  & 1 & 1 & 1 & 1 & 1 \\ 
  201635 & 12 &  & 0 & 0 & 0 & 0 & 0 \\ 
  201636 & 10 &  & 0 & 0 & 0 & 0 & 0 \\ 
  201637 & 12 &  & 0 & 0 & 0 & 0 & 0 \\ 
   \hline
\end{tabular}

\end{center}
\small{\hyperlink{vartab}{ver descrição das variáveis}}\end{minipage}\hfill\begin{minipage}[t]{.30\linewidth}
\begin{mdframed}[style=sidebar,frametitle={}]
\textbf{\hyperlink{municips}{Municipios}}\begin{itemize}\gsquare Mallet 
\gsquare Rio Azul 
\gsquare Fernandes Pinheiro 
\gsquare Guamiranga 
\gsquare Imbituva 
\gsquare Inácio Martins 
\gsquare Irati 
\gsquare Rebouças 
\gsquare Teixeira Soares 
\end{itemize}\BackToContents\end{mdframed}\hfill\end{minipage}\newpage\begin{minipage}[t]{.66\linewidth}
\hypertarget{PntG}{\heading{Regional Ponta Grossa}{6pt}}
\includegraphics[width=0.8\textwidth]{figs/MapaPR_PontaGrossa.png}\vspace{0.5cm}\vspace{0.5cm}\begin{center}
\captionof{figure}{Casos notificados de dengue e Índice de menção em midia social sobre dengue na Regional Ponta Grossa }\includegraphics[width=1\textwidth]{figs/tweetPR_PontaGrossa.png}\end{center}
\captionof{table}{Resumo das últimas seis semanas epidemiológicas na Regional Ponta Grossa }\begin{center}
% latex table generated in R 3.3.0 by xtable 1.8-2 package
% Tue Aug 30 16:39:56 2016
\begin{tabular}{c|ccccccc}
  \hline
SE & temperatura & tweet & casos notif & casos preditos & ICmin & ICmax & incidência \\ 
  \hline
201628 & 13 &  & 1 & 1 & 1 & 1 & 0 \\ 
  201629 & 6 & 1 & 1 & 1 & 1 & 1 & 0 \\ 
  201630 & 10 &  & 2 & 2 & 2 & 2 & 0 \\ 
  201631 & 11 &  & 0 & 0 & 0 & 0 & 0 \\ 
  201632 & 7 &  & 2 & 2 & 2 & 2 & 0 \\ 
  201633 & 11 &  & 1 & 1 & 1 & 1 & 0 \\ 
   \hline
\end{tabular}

\end{center}
\small{\hyperlink{vartab}{ver descrição das variáveis}}\end{minipage}\hfill\begin{minipage}[t]{.30\linewidth}
\begin{mdframed}[style=sidebar,frametitle={}]
\textbf{\hyperlink{municips}{Municipios}}\begin{itemize}\gsquare Sengés 
\gsquare Ipiranga 
\gsquare Ivaí 
\gsquare Jaguariaíva 
\gsquare Arapoti 
\gsquare Carambeí 
\gsquare Castro 
\gsquare Palmeira 
\gsquare Piraí do Sul 
\gsquare Ponta Grossa 
\gsquare Porto Amazonas 
\gsquare São João do Triunfo 
\end{itemize}\BackToContents\end{mdframed}\hfill\end{minipage}\newpage\begin{minipage}[t]{.66\linewidth}
\hypertarget{Crtb}{\heading{Regional Curitiba}{6pt}}
\includegraphics[width=0.8\textwidth]{figs/MapaPR_Curitiba.png}\vspace{0.5cm}\vspace{0.5cm}\begin{center}
\captionof{figure}{Casos notificados de dengue e Índice de menção em midia social sobre dengue na Regional Curitiba }\includegraphics[width=1\textwidth]{figs/tweetPR_Curitiba.png}\end{center}
\captionof{table}{Resumo das últimas seis semanas epidemiológicas na Regional Curitiba }\begin{center}
% latex table generated in R 3.3.0 by xtable 1.8-2 package
% Tue Aug 30 16:39:59 2016
\begin{tabular}{c|ccccccc}
  \hline
SE & temperatura & tweet & casos notif & casos preditos & ICmin & ICmax & incidência \\ 
  \hline
201628 & 13 &  & 34 & 37 & 34 & 38 & 1 \\ 
  201629 & 6 & 2 & 27 & 30 & 28 & 31 & 1 \\ 
  201630 & 10 &  & 36 & 43 & 38 & 44 & 1 \\ 
  201631 & 11 &  & 33 & 44 & 40 & 45 & 1 \\ 
  201632 & 7 &  & 18 & 27 & 22 & 28 & 1 \\ 
  201633 & 11 &  & 13 & 29 & 23 & 31 & 0 \\ 
   \hline
\end{tabular}

\end{center}
\small{\hyperlink{vartab}{ver descrição das variáveis}}\end{minipage}\hfill\begin{minipage}[t]{.30\linewidth}
\begin{mdframed}[style=sidebar,frametitle={}]
\textbf{\hyperlink{municips}{Municipios}}\begin{itemize}\gsquare Cerro Azul 
\gsquare Colombo 
\gsquare Contenda 
\gsquare Curitiba 
\gsquare Fazenda Rio Grande 
\gsquare Itaperuçu 
\gsquare Lapa 
\gsquare Mandirituba 
\gsquare Adrianópolis 
\gsquare Agudos do Sul 
\gsquare Almirante Tamandaré 
\gsquare Araucária 
\gsquare Balsa Nova 
\gsquare Bocaiúva do Sul 
\gsquare Campina Grande do Sul 
\gsquare Campo do Tenente 
\gsquare Campo Largo 
\gsquare Campo Magro 
\gsquare Piên 
\gsquare Pinhais 
\gsquare Piraquara 
\gsquare Quatro Barras 
\gsquare Quitandinha 
\gsquare Rio Branco do Sul 
\gsquare Rio Negro 
\gsquare São José dos Pinhais 
\gsquare Tijucas do Sul 
\gsquare Tunas do Paraná 
\gsquare Doutor Ulysses 
\end{itemize}\BackToContents\end{mdframed}\hfill\end{minipage}\newpage\begin{minipage}[t]{.66\linewidth}
\hypertarget{Prng}{\heading{Regional Paranaguá}{6pt}}
\includegraphics[width=0.8\textwidth]{figs/MapaPR_Paranagua.png}\vspace{0.5cm}\vspace{0.5cm}\begin{center}
\captionof{figure}{Casos notificados de dengue e Índice de menção em midia social sobre dengue na Regional Paranaguá }\includegraphics[width=1\textwidth]{figs/tweetPR_Paranagua.png}\end{center}
\captionof{table}{Resumo das últimas seis semanas epidemiológicas na Regional Paranaguá }\begin{center}
% latex table generated in R 3.3.0 by xtable 1.8-2 package
% Tue Aug 30 16:40:06 2016
\begin{tabular}{c|ccccccc}
  \hline
SE & temperatura & tweet & casos notif & casos preditos & ICmin & ICmax & incidência \\ 
  \hline
201628 & 13 &  & 25 & 25 & 25 & 25 & 9 \\ 
  201629 & 6 & 0 & 15 & 15 & 15 & 15 & 5 \\ 
  201630 & 10 &  & 10 & 10 & 10 & 10 & 4 \\ 
  201631 & 11 &  & 2 & 2 & 2 & 2 & 1 \\ 
  201632 & 7 &  & 2 & 2 & 2 & 2 & 1 \\ 
  201633 & 11 & 0 & 11 & 13 & 11 & 13 & 4 \\ 
   \hline
\end{tabular}

\end{center}
\small{\hyperlink{vartab}{ver descrição das variáveis}}\end{minipage}\hfill\begin{minipage}[t]{.30\linewidth}
\begin{mdframed}[style=sidebar,frametitle={}]
\textbf{\hyperlink{municips}{Municipios}}\begin{itemize}\gsquare Guaraqueçaba 
\gsquare Guaratuba 
\gsquare Matinhos 
\gsquare Antonina 
\gsquare Morretes 
\gsquare Paranaguá 
\gsquare Pontal do Paraná 
\end{itemize}\BackToContents\end{mdframed}\hfill\end{minipage}\newpage\begin{minipage}[t]{.66\linewidth}
\hypertarget{CrnP}{\heading{Regional Cornélio Procópio}{6pt}}
\includegraphics[width=0.8\textwidth]{figs/MapaPR_CornelioProcopio.png}\vspace{0.5cm}\vspace{0.5cm}\begin{center}
\captionof{figure}{Casos notificados de dengue e Índice de menção em midia social sobre dengue na Regional Cornélio Procópio }\includegraphics[width=1\textwidth]{figs/tweetPR_CornelioProcopio.png}\end{center}
\captionof{table}{Resumo das últimas seis semanas epidemiológicas na Regional Cornélio Procópio }\begin{center}
% latex table generated in R 3.3.1 by xtable 1.8-2 package
% Fri Oct  7 11:53:37 2016
\begin{tabular}{c|ccccccc}
  \hline
SE & temperatura & tweet & casos notif & casos preditos & ICmin & ICmax & incidência \\ 
  \hline
201634 & 12 &  & 2 & 2 & 2 & 2 & 1 \\ 
  201635 & 17 &  & 4 & 4 & 4 & 4 & 2 \\ 
  201636 & 13 &  & 0 & 0 & 0 & 0 & 0 \\ 
  201637 & 16 &  & 6 & 6 & 6 & 6 & 3 \\ 
  201638 & 15 &  & 2 & 2 & 2 & 2 & 1 \\ 
  201639 & 14 &  & 1 & 1 & 1 & 1 & 0 \\ 
   \hline
\end{tabular}

\end{center}
\small{\hyperlink{vartab}{ver descrição das variáveis}}\end{minipage}\hfill\begin{minipage}[t]{.30\linewidth}
\begin{mdframed}[style=sidebar,frametitle={}]
\textbf{\hyperlink{municips}{Municipios}}\begin{itemize}\gsquare Itambaracá 
\gsquare Andirá 
\gsquare Bandeirantes 
\gsquare Congonhinhas 
\gsquare Cornélio Procópio 
\gsquare Leópolis 
\gsquare Nova América da Colina 
\gsquare Nova Fátima 
\gsquare Nova Santa Bárbara 
\gsquare Rancho Alegre 
\gsquare Ribeirão do Pinhal 
\gsquare Santa Amélia 
\gsquare Santa Cecília do Pavão 
\gsquare Santa Mariana 
\gsquare Santo Antônio do Paraíso 
\gsquare São Jerônimo da Serra 
\gsquare São Sebastião da Amoreira 
\gsquare Sapopema 
\gsquare Sertaneja 
\gsquare Uraí 
\gsquare Abatiá 
\end{itemize}\BackToContents\end{mdframed}\hfill\end{minipage}\newpage\begin{minipage}[t]{.66\linewidth}
\hypertarget{Ivpr}{\heading{Regional Ivaiporã}{6pt}}
\includegraphics[width=0.8\textwidth]{figs/MapaPR_Ivaipora.png}\vspace{0.5cm}\vspace{0.5cm}\begin{center}
\captionof{figure}{Casos notificados de dengue e Índice de menção em midia social sobre dengue na Regional Ivaiporã }\includegraphics[width=1\textwidth]{figs/tweetPR_Ivaipora.png}\end{center}
\captionof{table}{Resumo das últimas seis semanas epidemiológicas na Regional Ivaiporã }\begin{center}
% latex table generated in R 3.3.1 by xtable 1.8-2 package
% Thu Sep 15 13:59:57 2016
\begin{tabular}{c|ccccccc}
  \hline
SE & temperatura & tweet & casos notif & casos preditos & ICmin & ICmax & incidência \\ 
  \hline
201631 & 15 &  & 1 & 1 & 1 & 1 & 1 \\ 
  201632 & 13 &  & 0 & 0 & 0 & 0 & 0 \\ 
  201633 & 16 & 0 & 0 & 0 & 0 & 0 & 0 \\ 
  201634 & 12 &  & 1 & 1 & 1 & 1 & 1 \\ 
  201635 & 17 &  & 1 & 1 & 1 & 1 & 1 \\ 
  201636 & 13 &  & 0 & 0 & 0 & 0 & 0 \\ 
   \hline
\end{tabular}

\end{center}
\small{\hyperlink{vartab}{ver descrição das variáveis}}\end{minipage}\hfill\begin{minipage}[t]{.30\linewidth}
\begin{mdframed}[style=sidebar,frametitle={}]
\textbf{\hyperlink{municips}{Municipios}}\begin{itemize}\gsquare Arapuã 
\gsquare Ariranha do Ivaí 
\gsquare Cândido de Abreu 
\gsquare Cruzmaltina 
\gsquare Godoy Moreira 
\gsquare Ivaiporã 
\gsquare Jardim Alegre 
\gsquare Lidianópolis 
\gsquare Lunardelli 
\gsquare Manoel Ribas 
\gsquare Mato Rico 
\gsquare Nova Tebas 
\gsquare Rio Branco do Ivaí 
\gsquare Rosário do Ivaí 
\gsquare Santa Maria do Oeste 
\gsquare São João do Ivaí 
\end{itemize}\BackToContents\end{mdframed}\hfill\end{minipage}\newpage\begin{minipage}[t]{.66\linewidth}
\hypertarget{Told}{\heading{Regional Toledo}{6pt}}
\includegraphics[width=0.8\textwidth]{figs/MapaPR_Toledo.png}\vspace{0.5cm}\vspace{0.5cm}\begin{center}
\captionof{figure}{Casos notificados de dengue e Índice de menção em midia social sobre dengue na Regional Toledo }\includegraphics[width=1\textwidth]{figs/tweetPR_Toledo.png}\end{center}
\captionof{table}{Resumo das últimas seis semanas epidemiológicas na Regional Toledo }\begin{center}
% latex table generated in R 3.3.0 by xtable 1.8-2 package
% Tue Aug 23 12:42:32 2016
\begin{tabular}{c|ccccccc}
  \hline
SE & temperatura & tweet & casos notif & casos preditos & ICmin & ICmax & incidência \\ 
  \hline
201628 & 16 &  & 17 & 17 & 17 & 17 & 4 \\ 
  201629 & 5 & 0 & 4 & 4 & 4 & 4 & 1 \\ 
  201630 & 13 &  & 12 & 12 & 12 & 12 & 3 \\ 
  201631 & 16 &  & 12 & 12 & 12 & 12 & 3 \\ 
  201632 & 11 &  & 1 & 1 & 1 & 1 & 0 \\ 
  201633 & 16 &  & 7 & 7 & 7 & 8 & 2 \\ 
   \hline
\end{tabular}

\end{center}
\small{\hyperlink{vartab}{ver descrição das variáveis}}\end{minipage}\hfill\begin{minipage}[t]{.30\linewidth}
\begin{mdframed}[style=sidebar,frametitle={}]
\textbf{\hyperlink{municips}{Municipios}}\begin{itemize}\gsquare Assis Chateaubriand 
\gsquare Diamante D'Oeste 
\gsquare Entre Rios do Oeste 
\gsquare Guaíra 
\gsquare Marechal Cândido Rondon 
\gsquare Maripá 
\gsquare Mercedes 
\gsquare Nova Santa Rosa 
\gsquare Ouro Verde do Oeste 
\gsquare Palotina 
\gsquare Pato Bragado 
\gsquare Quatro Pontes 
\gsquare Santa Helena 
\gsquare São José das Palmeiras 
\gsquare São Pedro do Iguaçu 
\gsquare Terra Roxa 
\gsquare Toledo 
\gsquare Tupãssi 
\end{itemize}\BackToContents\end{mdframed}\hfill\end{minipage}\newpage\begin{minipage}[t]{.66\linewidth}
\hypertarget{Jcrz}{\heading{Regional Jacarezinho}{6pt}}
\includegraphics[width=0.8\textwidth]{figs/MapaPR_Jacarezinho.png}\vspace{0.5cm}\vspace{0.5cm}\begin{center}
\captionof{figure}{Casos notificados de dengue e Índice de menção em midia social sobre dengue na Regional Jacarezinho }\includegraphics[width=1\textwidth]{figs/tweetPR_Jacarezinho.png}\end{center}
\captionof{table}{Resumo das últimas seis semanas epidemiológicas na Regional Jacarezinho }\begin{center}
% latex table generated in R 3.3.1 by xtable 1.8-2 package
% Fri Oct  7 11:53:48 2016
\begin{tabular}{c|ccccccc}
  \hline
SE & temperatura & tweet & casos notif & casos preditos & ICmin & ICmax & incidência \\ 
  \hline
201634 & 12 &  & 1 & 1 & 1 & 1 & 0 \\ 
  201635 & 17 &  & 3 & 3 & 3 & 3 & 1 \\ 
  201636 & 13 &  & 1 & 1 & 1 & 1 & 0 \\ 
  201637 & 16 &  & 2 & 2 & 2 & 2 & 1 \\ 
  201638 & 15 &  & 3 & 3 & 3 & 4 & 1 \\ 
  201639 & 14 &  & 6 & 10 & 7 & 12 & 2 \\ 
   \hline
\end{tabular}

\end{center}
\small{\hyperlink{vartab}{ver descrição das variáveis}}\end{minipage}\hfill\begin{minipage}[t]{.30\linewidth}
\begin{mdframed}[style=sidebar,frametitle={}]
\textbf{\hyperlink{municips}{Municipios}}\begin{itemize}\gsquare Barra do Jacaré 
\gsquare Cambará 
\gsquare Carlópolis 
\gsquare Conselheiro Mairinck 
\gsquare Figueira 
\gsquare Guapirama 
\gsquare Ibaiti 
\gsquare Jaboti 
\gsquare Jacarezinho 
\gsquare Japira 
\gsquare Joaquim Távora 
\gsquare Jundiaí do Sul 
\gsquare Pinhalão 
\gsquare Quatiguá 
\gsquare Ribeirão Claro 
\gsquare Salto do Itararé 
\gsquare Santana do Itararé 
\gsquare Santo Antônio da Platina 
\gsquare São José da Boa Vista 
\gsquare Siqueira Campos 
\gsquare Tomazina 
\gsquare Wenceslau Braz 
\end{itemize}\BackToContents\end{mdframed}\hfill\end{minipage}\newpage\begin{minipage}[t]{.66\linewidth}
\hypertarget{TlmB}{\heading{Regional Telêmaco Borba}{6pt}}
\includegraphics[width=0.8\textwidth]{figs/MapaPR_TelemacoBorba.png}\vspace{0.5cm}\vspace{0.5cm}\begin{center}
\captionof{figure}{Casos notificados de dengue e Índice de menção em midia social sobre dengue na Regional Telêmaco Borba }\includegraphics[width=1\textwidth]{figs/tweetPR_TelemacoBorba.png}\end{center}
\captionof{table}{Resumo das últimas seis semanas epidemiológicas na Regional Telêmaco Borba }\begin{center}
% latex table generated in R 3.3.1 by xtable 1.8-2 package
% Mon Oct 31 13:05:00 2016
\begin{tabular}{c|ccccccc}
  \hline
SE & temperatura & tweet & casos notif & casos preditos & ICmin & ICmax & incidência \\ 
  \hline
201637 & 16 &  & 3 & 3 & 3 & 3 & 2 \\ 
  201638 & 15 &  & 1 & 1 & 1 & 1 & 1 \\ 
  201639 & 14 &  & 2 & 2 & 2 & 2 & 1 \\ 
  201640 & 15 &  & 1 & 1 & 1 & 1 & 1 \\ 
  201641 & 17 &  & 0 & 0 & 0 & 0 & 0 \\ 
  201642 & 23 &  & 1 & 1 & 1 & 1 & 1 \\ 
   \hline
\end{tabular}

\end{center}
\small{\hyperlink{vartab}{ver descrição das variáveis}}\end{minipage}\hfill\begin{minipage}[t]{.30\linewidth}
\begin{mdframed}[style=sidebar,frametitle={}]
\textbf{\hyperlink{municips}{Municipios}}\begin{itemize}\gsquare Curiúva 
\gsquare Imbaú 
\gsquare Ortigueira 
\gsquare Reserva 
\gsquare Telêmaco Borba 
\gsquare Tibagi 
\gsquare Ventania 
\end{itemize}\BackToContents\end{mdframed}\hfill\end{minipage}\newpage

%-----------------------------------------------------------------------------------%	MAIN BODY - THIRD PAGE
%-----------------------------------------------------------------------------------
 %\begin{minipage}[t]{1\linewidth} % Mini page taking up 100% of the actual page

 
      \hypertarget{municips}{\heading{Resumo da situação epidemiológica da dengue nos municípios de Paraná, na semana 38 de 2016}{6pt}}

      
     \begin{center}
            % latex table generated in R 3.3.0 by xtable 1.8-2 package
% Tue Aug 30 15:00:01 2016
\begin{longtable}{l|lllllll}
  \hline
Municipio & Regional & Temperatura & Tweets & Casos & Incidencia & Rt & Nivel \\ 
  \hline
\endhead
\hline
{\footnotesize Continua na próxima página}
\endfoot
\endlastfoot
Anahy & Cascavel & 16.3 & 0 & 0 & 0.0 & 0.0 & verde \\ 
  Boa Vista da Aparecida & Cascavel & 16.3 & 0 & 0 & 0.0 & 0.0 & verde \\ 
  Braganey & Cascavel & 16.3 & 0 & 0 & 0.0 & 0.0 & verde \\ 
  Cafelândia & Cascavel & 16.3 & 0 & 0 & 0.0 & 0.0 & verde \\ 
  Campo Bonito & Cascavel & 16.3 & 0 & 0 & 0.0 & 0.0 & verde \\ 
  Capitão Leônidas Marques & Cascavel & 16.3 & 0 & 0 & 0.0 & 0.0 & verde \\ 
  Cascavel & Cascavel & 16.3 & 0 & 12 & 5.5 & 1.1 & verde \\ 
  Catanduvas & Cascavel & 16.3 & 0 & 0 & 0.0 & 0.0 & verde \\ 
  Céu Azul & Cascavel & 16.3 & 0 & 0 & 0.0 & 0.0 & verde \\ 
  Corbélia & Cascavel & 16.3 & 0 & 1 & 5.9 & 2.9 & verde \\ 
  Diamante do Sul & Cascavel & 16.3 & 0 & 0 & 0.0 & 0.0 & verde \\ 
  Espigão Alto do Iguaçu & Cascavel & 16.3 & 0 & 0 & 0.0 & 0.0 & verde \\ 
  Formosa do Oeste & Cascavel & 16.3 & 0 & 0 & 0.0 & 0.0 & verde \\ 
  Guaraniaçu & Cascavel & 16.3 & 0 & 0 & 0.0 & 0.0 & verde \\ 
  Ibema & Cascavel & 16.3 & 0 & 0 & 0.0 & 0.0 & verde \\ 
  Iguatu & Cascavel & 16.3 & 0 & 0 & 0.0 & 0.0 & verde \\ 
  Iracema do Oeste & Cascavel & 16.3 & 0 & 0 & 0.0 & 0.0 & verde \\ 
  Jesuítas & Cascavel & 16.3 & 0 & 0 & 0.0 & 0.0 & verde \\ 
  Lindoeste & Cascavel & 16.3 & 0 & 1 & 19.1 & 12.1 & verde \\ 
  Nova Aurora & Cascavel & 16.3 & 0 & 0 & 0.0 & 0.0 & verde \\ 
  Quedas do Iguaçu & Cascavel & 16.3 & 0 & 1 & 3.1 & 12.1 & verde \\ 
  Santa Lúcia & Cascavel & 16.3 & 0 & 0 & 0.0 & 0.0 & verde \\ 
  Santa Tereza do Oeste & Cascavel & 16.3 & 0 & 0 & 0.0 & 0.0 & verde \\ 
  Três Barras do Paraná & Cascavel & 16.3 &  & 0 & 0.0 & 0.0 & verde \\ 
  Vera Cruz do Oeste & Cascavel & 16.3 &  & 1 & 11.1 & 2.5 & verde \\ 
  \hline
\end{longtable}

     \end{center}


      \BackToContents % Link back to the contents of the newsletter


\newpage

%---------------------------------------------------------------------------------
%	Variáves nas Tabelas, Créditos e Contato
%---------------------------------------------------------------------------------

\begin{minipage}[t]{1\linewidth} 

\hypertarget{vartab}{\heading{Lista das variáveis apresentadas nas tabelas:}{6pt}}

\begin{description}
\item [SE =] semana epidemiológica
\item [tweet =] número de tweets indicativos de casos de dengue na cidade
\item [temperatura =] média das temperaturas mínimas da semana
\item [casos notif =] casos notificados de dengue 
\item [casos preditos =] número de casos estimados após correção pelo atraso de notificação
\item [ICmin =] número mínimo de casos estimados (IC 95\%)
\item [ICmax =] número máximo de casos estimados (IC 95\%)
\item [Rt] número reprodutivo efetivo ($>$ 1 indica aumento de casos transmissão)
\item [p(Rt1) =] probabilidade do número reprodutivo ser maior que 1 ($>0.95$ indica aumento significativo de casos)
\item [inc =] incidência por 100.000 habitantes
\item [Nivel =] cor do alerta (verde, amarelo, laranja, vermelho)
\end{description}

\hypertarget{notas}{\heading{Notas}{6pt}}

\begin{itemize}
\item Os dados do sinan mais recentes ainda não foram totalmente digitados. Estimamos o número esperado de casos notificados considerando o tempo até os casos serem digitados.
\item Os dados de tweets são gerados pelo Observatório de Dengue (UFMG). Os tweets são processados para exclusão de informes e outros temas relacionados a dengue.
\item Algumas vezes, os casos da última semana ainda não estao disponíveis, nesse caso, usa-se uma estimação com base na tendência de variação da série.
\end{itemize}

\hypertarget{creditos}{\heading{Créditos}{6pt}}

Este é um projeto desenvolvido com apoio da SVS/MS em parceria com:

\begin{itemize}
\item Programa de Computação Científica, Fundação Oswaldo Cruz, Rio de Janeiro.
\item Escola de Matemática Aplicada, Fundação Getúlio Vargas.
\item Secretarias do Estado e Município do Rio de Janeiro.
\item Observatório de Dengue da UFMG
\item Secretaria Estadual de Saúde do Paraná.
\end{itemize}

      \BackToContents % Link back to the contents of the newsletter

\vspace{1cm}

\hrule
Para mais detalhes sobre o sistema de alerta InfoDengue, consultar: \url{http://info.dengue.mat.br}\\

\textbf{Contato}: \href{alerta\_dengue@fiocruz.br}{\nolinkurl{alerta\_dengue@fiocruz.br} }
\end{minipage} % fim da pagina de creditos

\end{document} 
