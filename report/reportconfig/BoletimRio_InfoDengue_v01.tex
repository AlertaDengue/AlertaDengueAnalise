%%%%%%%%%%%%%%%%%%%%%%%%%%%%%%%%%%%%%%%%%
% Boletim semanal municipal do InfoDengue
% Adapted by Claudia Codeco (Jun 2016)
%%%%%%%%%%%%%%%%%%%%%%%%%%%%%%%%%%%%%%%%%

\documentclass[10pt]{article} % The default font size is 10pt; 11pt and 12pt are alternatives

%%%%%%%%%%%%%%%%%%%%%%%%%%%%%%%%%%%%%%%%%
% Professional Newsletter Template
% Structural Definitions File
% Version 1.0 (09/03/14)
%
% Created by:
% Vel (vel@latextemplates.com)
% 
% This file has been downloaded from:
% http://www.LaTeXTemplates.com
%
% License:
% CC BY-NC-SA 3.0 (http://creativecommons.org/licenses/by-nc-sa/3.0/)
%
%%%%%%%%%%%%%%%%%%%%%%%%%%%%%%%%%%%%%%%%%

%----------------------------------------------------------------------------------------
%	REQUIRED PACKAGES
%----------------------------------------------------------------------------------------

\usepackage{graphicx} % Required for including images
\usepackage[utf8]{inputenc}
\usepackage{microtype} % Improved typography
\usepackage{multicol} % Used for the two-column layout of the document
\usepackage{booktabs} % Required for nice horizontal rules in tables
\usepackage{wrapfig} % Required for in-line images
\usepackage{float} % Required for forcing figures not to float with the [H] 
\usepackage{longtable}
\usepackage{fancyhdr} % footer 
%parameter
\usepackage{Sweave}

%-------------

% -------------
% footer style
\pagestyle{fancy}
\fancyhead{}
\fancyfoot{}
\renewcommand{\headrulewidth}{0.0pt}
\renewcommand{\footrulewidth}{0.4pt}

%------------------------------------------------
% Fonts

\usepackage{charter} % Use the Charter font as the main document font
\usepackage{courier} % Use the Courier font for \texttt (monospaced) only
%\usepackage[T1]{fontenc} % Use T1 font encoding
\usepackage[utf8]{inputenc}
\usepackage[brazilian]{babel}

%------------------------------------------------
% List Separation

\usepackage{enumitem} % Required to customize the list environments
\setlist{noitemsep,nolistsep} % Remove spacing before, after and within lists for a compact look

%------------------------------------------------
% Figure and Table Caption Styles

\usepackage{caption} % Required for changing caption styles
\captionsetup[table]{labelfont={bf,sf},labelsep=period,justification=justified} % Specify the table caption style
\captionsetup[figure]{labelfont={sf,bf},labelsep=period,justification=justified, font=small} % Specify the figure caption style
\setlength{\abovecaptionskip}{10pt} % Whitespace above captions

%------------------------------------------------
% Spacing Between Paragraphs

\makeatletter
\usepackage{parskip}
\setlength{\parskip}{6pt}
\newcommand{\@minipagerestore}{\setlength{\parskip}{6pt}}
\makeatother

%----------------------------------------------------------------------------------------
%	PAGE MARGINS AND SPACINGS
%----------------------------------------------------------------------------------------

\textwidth = 7 in % Text width
\textheight = 9.5 in % Text height
\oddsidemargin = -10pt % Left side margin on odd pages
\evensidemargin = -10pt % Left side margin on even pages
\topmargin = -10pt % Top margin
\headheight = 0pt % Remove the header by setting its space to 0
\headsep = 0pt % Remove the space between the header and top of the page
\parskip = 4pt % Space between paragraph
\parindent = 0.0in % Paragraph indentation
%\pagestyle{empty} % Disable page numbering

%----------------------------------------------------------------------------------------
%	COLORS
%----------------------------------------------------------------------------------------

\usepackage[dvipsnames,svgnames]{xcolor} % Required to specify custom colors
\definecolor{altncolor}{rgb}{.0,0.0,0.8} % Dark blue
\definecolor{myred}{rgb}{.8,0,0} % Dark red
\definecolor{myorange}{rgb}{.93,.57,.0} % Dark orange
\definecolor{myyellow}{rgb}{.93,.93,.0} % yellow
\definecolor{mygreen}{rgb}{.0,.55,.27} % dark green
\definecolor{mywhite}{rgb}{1,1,1} % dark green

\usepackage[colorlinks=true, linkcolor=altncolor, anchorcolor=altncolor, citecolor=altncolor, filecolor=altncolor, menucolor=altncolor, urlcolor=altncolor]{hyperref} % Use the color defined above for all links

% --------------------------------------
% Bullet styles
\usepackage{pifont}
\newcommand{\gsquare}{\item[\color{mygreen}\ding{110}]} 
\newcommand{\ysquare}{\item[\color{myyellow}\ding{110}]} 
\newcommand{\osquare}{\item[\color{myorange}\ding{110}]} 
\newcommand{\rsquare}{\item[\color{myred}\ding{110}]} 
\newcommand{\wsquare}{\item[\color{mywhite}\ding{110}]} 


%----------------------------------------------------------------------------------------
%	BOX STYLES
%----------------------------------------------------------------------------------------

\usepackage[framemethod=TikZ]{mdframed}% Required for creating boxes
\mdfdefinestyle{sidebar}{
    linecolor=black, % Outer line color
    outerlinewidth=0.5pt, % Outer line width
    roundcorner=0pt, % Amount of corner rounding
    innertopmargin=10pt, % Top margin
    innerbottommargin=10pt, % Bottom margin
    innerrightmargin=10pt, % Right margin
    innerleftmargin=10pt, % Left margin
    backgroundcolor=white, % Box background color
    frametitlebackgroundcolor=white, % Title background color
    frametitlerule=false, % Title rule - true or false
    frametitlerulecolor=white, % Title rule color
    frametitlerulewidth=0.5pt, % Title rule width
    frametitlefont=\Large, % Title heading font specification
    font=\small
}

\mdfdefinestyle{intextbox}{
    linecolor=blue, % Outer line color
    outerlinewidth=0.5pt, % Outer line width
    roundcorner=10pt, % Amount of corner rounding
    innertopmargin=7pt, % Top margin
    innerbottommargin=7pt, % Bottom margin
    innerrightmargin=7pt, % Right margin
    innerleftmargin=7pt, % Left margin
    backgroundcolor=white, % Box background color
    frametitlebackgroundcolor=white, % Title background color
    frametitlerule=false, % Title rule - true or false
    frametitlerulecolor=white, % Title rule color
    frametitlerulewidth=0.5pt, % Title rule width
    frametitlefont=\Large % Title heading font specification
}

%----------------------------------------------------------------------------------------
%	HEADING STYLE
%----------------------------------------------------------------------------------------

\newcommand{\heading}[2]{ % Define the \heading command
%\vspace{#2} % White space above the heading
{\begin{center}\Large\textbf{#1}\end{center}} % The heading style
%\vspace{#2} % White space below the heading
}

%\newcommand{\Inicio}{\hyperlink{contents}{{\small Inicio}}} % Define a command for linking back to the contents of the newsletter
\newcommand{\BackToContents}{\hyperlink{contents}{{\small Início}}} % Include the document which specifies all packages and structural customizations for this template

% Carregando dados  (atualmente é manual)

\fancyfoot[C]{Boletim Municipal - Rio de Janeiro}
\fancyfoot[R]{SE 14 de 2016}
\fancyfoot[L]{\href{http://info.dengue.mat.br}{InfoDengue}}

\usepackage{Sweave}
\begin{document}
\input{BoletimRio_InfoDengue_v01-concordance}

%---------------------------------------------------------------------------------
%	HEADER IMAGE
%---------------------------------------------------------------------------------

\begin{figure}[H]
\centering\includegraphics[width=1\linewidth]{InfoDengue2.png}  
\end{figure}

\centerline {\color{altncolor}\rule{1\linewidth}{2.75pt}} % Horizontal line

%---------------------------------------------------------------------------------
%	CAIXA LATERAL - PRIMEIRA PAGINA
%--------------------------------------------------------------------------------

\begin{minipage}[t]{.30\linewidth} % Mini page taking up 30% of the actual page
\begin{mdframed}[style=sidebar,frametitle={}] % Sidebar box

%-----------------------------------------------------------
\textbf{{\LARGE{Rio de Janeiro}}}

\hypertarget{contents}{\textbf{{\large Boletim Semanal}}} 

\textbf{Semana 14 de 2016} % se, ano, em pp.RData

\begin{itemize}
\item \hyperlink{estado}{O Estado} 
\item \hyperlink{regional}{A Regional de Saúde}
\item \hyperlink{municipio}{O Município}
\end{itemize}

\textbf{Áreas Programáticas da Saúde}
\begin{itemize}
\item \hyperlink{ap1}{APS 1.0} 
\item \hyperlink{ap21}{APS 2.1} 
\item \hyperlink{ap22}{APS 2.2} 
\item \hyperlink{ap31}{APS 3.1} 
\item \hyperlink{ap32}{APS 3.2} 
\item \hyperlink{ap33}{APS 3.3} 
\item \hyperlink{ap4}{APS 4.0} 
\item \hyperlink{ap51}{APS 5.1} 
\item \hyperlink{ap52}{APS 5.2} 
\item \hyperlink{ap53}{APS 5.3} 
\end{itemize}

\centerline {\rule{.75\linewidth}{.25pt}} % Horizontal line

%-----------------------------------------------------------

\hyperlink{vartab}{Variáveis nas Tabelas}

\hyperlink{notas}{Notas} % These link to their appropriate sections in the newsletter

\hyperlink{creditos}{Créditos} % These link to their appropriate sections in the newsletter

\centerline {\rule{.75\linewidth}{.25pt}} % Horizontal line

\textbf{Contato}
\begin{description}
\item \href{mailto:alerta\_dengue@fiocruz.br}{alerta\_dengue@fiocruz.br}  
\end{description}

\end{mdframed}
\end{minipage}\hfill % End the sidebar mini page 
\begin{minipage}[t]{.66\linewidth} % Nao pode colocar espaco acima senao ela nao fica lado a lado

%--------------------------------------------------------------------------------
%	TEXTO PRINCIPAL - PRIMEIRA PAGINA - ALERTA A NIVEL DO ESTADO
%-------------------------------------------------------------------------------
\hypertarget{estado}{\heading{Situação da Dengue no Estado do Rio de Janeiro}{6pt}} % \hypertarget provides a label to reference using 

Desde o início do ano, 50707 casos foram registrados no estado, sendo 1983 na última semana. A figura abaixo mostra as condições de transmissão em cada município.

% Mapa estadual
\includegraphics[width=0.8\textwidth]{../RJ/figs/Mapa_ERJ.png}
%-----------------------------------------------------------

Dos 92 municipios, 8 encontram-se em nivel verde, 72 em nivel amarelo, 3 em nivel laranja e 9 em nivel vermelho referentes a semana epidemiológica 14-2016. Para informações mais atualizadas
sobre o município do Rio de Janeiro ter acesso ao mapa interativo do estado consultar em \href{http://info.dengue.mat.br}{\textit{Info Dengue}}.


%--------------------------------------------------------------------------------
%	PRIMEIRA PAGINA : BOX - CODIGO DE CORES
%--------------------------------------------------------------------------------
\vspace{1cm}
\begin{mdframed}[style=intextbox,frametitle={}] % Sidebar box

\hypertarget{descriptivebox}{\heading{O código de Cores}{1pt}} % \hypertarget provides a label to reference using \hyperlink{label}{link text}
As cores indicam niveis de atenção
\begin{description}
\item[Verde:] temperaturas amenas, baixa incidência de casos.      
\item[Amarelo:] temperatura propícia para a população do vetor e transmissão da dengue.
\item[Laranja:] transmissão aumentada e sustentada de dengue. 
\item[Vermelho:] incidência alta de dengue, acima dos 90\% históricos.
\end{description}
\end{mdframed}

\textbf{Na semana passada:} 65 municípios em nivel amarelo, 7 em laranja e 12 em vermelho.  

\end{minipage} % End the main body - first page mini page


%----------------------------------------------------------------------------------
%	SEGUNDA PAGINA: TEXTO PRINCIPAL - LOOP PELAS REGIONAIS
%----------------------------------------------------------------------------------

\begin{minipage}[t]{.66\linewidth}\hypertarget{MtrI}{\heading{Situação da Dengue na Regional NA}{6pt}}\includegraphics[\width=8\textwidth]{/home/administrador/Repositorios/AlertaDengueAnalise/report/RJ/figs/MapaRJ_NA.png}\captionof{table}[tabregional]{Resumo das últimas seis semanas epidemiológicas na Regional NA }\begin{center}\input{/home/administrador/Repositorios/AlertaDengueAnalise/report/RJ/figs/tabregionalRJ_NA.tex}\end{center}\small{\hyperlink{vartab}{ver descrição das variáveis}}\begin{center}\captionof{figure}{Casos notificados de dengue e Índice de menção em midia social sobre dengue na Regional NA }\includegraphics[width=1.3\textwidth]{/home/administrador/Repositorios/AlertaDengueAnalise/report/RJ/figs/tweetRJ_NA.png}\end{center}\end{minipage}\hfill\begin{minipage}[t]{.30\linewidth}\begin{mdframed}[style=sidebar,frametitle={}]\textbf{\hyperlink{municips}{Municipios}}\begin{itemize}\wsquare NA 
\wsquare NA 
\wsquare NA 
\wsquare NA 
\wsquare NA 
\wsquare NA 
\wsquare NA 
\wsquare NA 
\wsquare NA 
\wsquare NA 
\wsquare NA 
\wsquare NA 
\wsquare NA 
\wsquare NA 
\wsquare NA 
\wsquare NA 
\wsquare NA 
\wsquare NA 
\wsquare NA 
\wsquare NA 
\wsquare NA 
\wsquare NA 
\wsquare NA 
\wsquare NA 
\wsquare NA 
\wsquare NA 
\wsquare NA 
\wsquare NA 
\wsquare NA 
\wsquare NA 
\wsquare NA 
\wsquare NA 
\wsquare NA 
\wsquare NA 
\wsquare NA 
\wsquare NA 
\wsquare NA 
\wsquare NA 
\wsquare NA 
\wsquare NA 
\wsquare NA 
\wsquare NA 
\wsquare NA 
\wsquare NA 
\wsquare NA 
\wsquare NA 
\wsquare NA 
\wsquare NA 
\wsquare NA 
\wsquare NA 
\wsquare NA 
\wsquare NA 
\wsquare NA 
\wsquare NA 
\wsquare NA 
\wsquare NA 
\wsquare NA 
\wsquare NA 
\wsquare NA 
\wsquare NA 
\wsquare NA 
\wsquare NA 
\wsquare NA 
\wsquare NA 
\wsquare NA 
\wsquare NA 
\wsquare NA 
\wsquare NA 
\wsquare NA 
\wsquare NA 
\wsquare NA 
\wsquare NA 
\wsquare NA 
\wsquare NA 
\wsquare NA 
\wsquare NA 
\wsquare NA 
\wsquare NA 
\wsquare NA 
\wsquare NA 
\wsquare NA 
\wsquare NA 
\wsquare NA 
\wsquare NA 
\wsquare NA 
\wsquare NA 
\wsquare NA 
\wsquare NA 
\wsquare NA 
\wsquare NA 
\wsquare NA 
\wsquare NA \end{itemize}\BackToContents\end{mdframed}\hfill\end{minipage}\newpage