%%%%%%%%%%%%%%%%%%%%%%%%%%%%%%%%%%%%%%%%%
% Boletim semanal municipal do InfoDengue
% Adapted by Claudia Codeco (Jun 2016)
%%%%%%%%%%%%%%%%%%%%%%%%%%%%%%%%%%%%%%%%%

\documentclass[10pt]{article} % The default font size is 10pt; 11pt and 12pt are alternatives

%%%%%%%%%%%%%%%%%%%%%%%%%%%%%%%%%%%%%%%%%
% Professional Newsletter Template
% Structural Definitions File
% Version 1.0 (09/03/14)
%
% Created by:
% Vel (vel@latextemplates.com)
% 
% This file has been downloaded from:
% http://www.LaTeXTemplates.com
%
% License:
% CC BY-NC-SA 3.0 (http://creativecommons.org/licenses/by-nc-sa/3.0/)
%
%%%%%%%%%%%%%%%%%%%%%%%%%%%%%%%%%%%%%%%%%

%----------------------------------------------------------------------------------------
%	REQUIRED PACKAGES
%----------------------------------------------------------------------------------------

\usepackage{graphicx} % Required for including images
\usepackage[utf8]{inputenc}
\usepackage{microtype} % Improved typography
\usepackage{multicol} % Used for the two-column layout of the document
\usepackage{booktabs} % Required for nice horizontal rules in tables
\usepackage{wrapfig} % Required for in-line images
\usepackage{float} % Required for forcing figures not to float with the [H] 
\usepackage{longtable}
\usepackage{fancyhdr} % footer 
%parameter
\usepackage{Sweave}

%-------------

% -------------
% footer style
\pagestyle{fancy}
\fancyhead{}
\fancyfoot{}
\renewcommand{\headrulewidth}{0.0pt}
\renewcommand{\footrulewidth}{0.4pt}

%------------------------------------------------
% Fonts

\usepackage{charter} % Use the Charter font as the main document font
\usepackage{courier} % Use the Courier font for \texttt (monospaced) only
%\usepackage[T1]{fontenc} % Use T1 font encoding
\usepackage[utf8]{inputenc}
\usepackage[brazilian]{babel}

%------------------------------------------------
% List Separation

\usepackage{enumitem} % Required to customize the list environments
\setlist{noitemsep,nolistsep} % Remove spacing before, after and within lists for a compact look

%------------------------------------------------
% Figure and Table Caption Styles

\usepackage{caption} % Required for changing caption styles
\captionsetup[table]{labelfont={bf,sf},labelsep=period,justification=justified} % Specify the table caption style
\captionsetup[figure]{labelfont={sf,bf},labelsep=period,justification=justified, font=small} % Specify the figure caption style
\setlength{\abovecaptionskip}{10pt} % Whitespace above captions

%------------------------------------------------
% Spacing Between Paragraphs

\makeatletter
\usepackage{parskip}
\setlength{\parskip}{6pt}
\newcommand{\@minipagerestore}{\setlength{\parskip}{6pt}}
\makeatother

%----------------------------------------------------------------------------------------
%	PAGE MARGINS AND SPACINGS
%----------------------------------------------------------------------------------------

\textwidth = 7 in % Text width
\textheight = 9.5 in % Text height
\oddsidemargin = -10pt % Left side margin on odd pages
\evensidemargin = -10pt % Left side margin on even pages
\topmargin = -10pt % Top margin
\headheight = 0pt % Remove the header by setting its space to 0
\headsep = 0pt % Remove the space between the header and top of the page
\parskip = 4pt % Space between paragraph
\parindent = 0.0in % Paragraph indentation
%\pagestyle{empty} % Disable page numbering

%----------------------------------------------------------------------------------------
%	COLORS
%----------------------------------------------------------------------------------------

\usepackage[dvipsnames,svgnames]{xcolor} % Required to specify custom colors
\definecolor{altncolor}{rgb}{.0,0.0,0.8} % Dark blue
\definecolor{myred}{rgb}{.8,0,0} % Dark red
\definecolor{myorange}{rgb}{.93,.57,.0} % Dark orange
\definecolor{myyellow}{rgb}{.93,.93,.0} % yellow
\definecolor{mygreen}{rgb}{.0,.55,.27} % dark green
\definecolor{mywhite}{rgb}{1,1,1} % dark green

\usepackage[colorlinks=true, linkcolor=altncolor, anchorcolor=altncolor, citecolor=altncolor, filecolor=altncolor, menucolor=altncolor, urlcolor=altncolor]{hyperref} % Use the color defined above for all links

% --------------------------------------
% Bullet styles
\usepackage{pifont}
\newcommand{\gsquare}{\item[\color{mygreen}\ding{110}]} 
\newcommand{\ysquare}{\item[\color{myyellow}\ding{110}]} 
\newcommand{\osquare}{\item[\color{myorange}\ding{110}]} 
\newcommand{\rsquare}{\item[\color{myred}\ding{110}]} 
\newcommand{\wsquare}{\item[\color{mywhite}\ding{110}]} 


%----------------------------------------------------------------------------------------
%	BOX STYLES
%----------------------------------------------------------------------------------------

\usepackage[framemethod=TikZ]{mdframed}% Required for creating boxes
\mdfdefinestyle{sidebar}{
    linecolor=black, % Outer line color
    outerlinewidth=0.5pt, % Outer line width
    roundcorner=0pt, % Amount of corner rounding
    innertopmargin=10pt, % Top margin
    innerbottommargin=10pt, % Bottom margin
    innerrightmargin=10pt, % Right margin
    innerleftmargin=10pt, % Left margin
    backgroundcolor=white, % Box background color
    frametitlebackgroundcolor=white, % Title background color
    frametitlerule=false, % Title rule - true or false
    frametitlerulecolor=white, % Title rule color
    frametitlerulewidth=0.5pt, % Title rule width
    frametitlefont=\Large, % Title heading font specification
    font=\small
}

\mdfdefinestyle{intextbox}{
    linecolor=blue, % Outer line color
    outerlinewidth=0.5pt, % Outer line width
    roundcorner=10pt, % Amount of corner rounding
    innertopmargin=7pt, % Top margin
    innerbottommargin=7pt, % Bottom margin
    innerrightmargin=7pt, % Right margin
    innerleftmargin=7pt, % Left margin
    backgroundcolor=white, % Box background color
    frametitlebackgroundcolor=white, % Title background color
    frametitlerule=false, % Title rule - true or false
    frametitlerulecolor=white, % Title rule color
    frametitlerulewidth=0.5pt, % Title rule width
    frametitlefont=\Large % Title heading font specification
}

%----------------------------------------------------------------------------------------
%	HEADING STYLE
%----------------------------------------------------------------------------------------

\newcommand{\heading}[2]{ % Define the \heading command
%\vspace{#2} % White space above the heading
{\begin{center}\Large\textbf{#1}\end{center}} % The heading style
%\vspace{#2} % White space below the heading
}

%\newcommand{\Inicio}{\hyperlink{contents}{{\small Inicio}}} % Define a command for linking back to the contents of the newsletter
\newcommand{\BackToContents}{\hyperlink{contents}{{\small Início}}} % Include the document which specifies all packages and structural customizations for this template

% Carregando dados  (atualmente é manual)

\fancyfoot[C]{Boletim Municipal - Rio de Janeiro}
\fancyfoot[R]{SE 27 de 2016}
\fancyfoot[L]{\href{http://info.dengue.mat.br}{InfoDengue}}

\usepackage{Sweave}
\begin{document}
\input{BoletimRio_InfoDengue_v01-concordance}

%---------------------------------------------------------------------------------
%	HEADER IMAGE
%---------------------------------------------------------------------------------

\begin{figure}[H]
\centering\includegraphics[width=1\linewidth]{InfoDengue2.png}  
\end{figure}

\centerline {\color{altncolor}\rule{1\linewidth}{2.75pt}} % Horizontal line

%---------------------------------------------------------------------------------
%	CAIXA LATERAL - PRIMEIRA PAGINA
%--------------------------------------------------------------------------------

\begin{minipage}[t]{.30\linewidth} % Mini page taking up 30% of the actual page
\begin{mdframed}[style=sidebar,frametitle={}] % Sidebar box

%-----------------------------------------------------------
\textbf{{\LARGE{Rio de Janeiro}}}

\hypertarget{contents}{\textbf{{\large Boletim Semanal}}} 

\textbf{Semana 27 de 2016} % se, ano, em pp.RData

\begin{itemize}
 \item \hyperlink{estado}{O Estado} 
 \item \hyperlink{regional}{A Regional Metropolitana I}
 \item \hyperlink{mun}{O Município}
 \item \hyperlink{aps}{Áreas Programáticas da Saúde}
\end{itemize}
\centerline {\rule{.75\linewidth}{.25pt}} % Horizontal line

%-----------------------------------------------------------

\hyperlink{vartab}{Variáveis nas Tabelas}

\hyperlink{notas}{Notas} % These link to their appropriate sections in the newsletter

\hyperlink{creditos}{Créditos} % These link to their appropriate sections in the newsletter

\centerline {\rule{.75\linewidth}{.25pt}} % Horizontal line

\textbf{Contato}
\begin{description}
\item \href{mailto:alerta\_dengue@fiocruz.br}{alerta\_dengue@fiocruz.br}  
\end{description}

\end{mdframed}
\end{minipage}\hfill % End the sidebar mini page 
\begin{minipage}[t]{.66\linewidth} % Nao pode colocar espaco acima senao ela nao fica lado a lado

%--------------------------------------------------------------------------------
%	TEXTO PRINCIPAL - PRIMEIRA PAGINA - ALERTA A NIVEL DO ESTADO
%-------------------------------------------------------------------------------
\hypertarget{estado}{\heading{Situação da Dengue no Estado do Rio de Janeiro}{6pt}} % \hypertarget provides a label to reference using 

Desde o início do ano, 93188 casos foram registrados no estado, sendo 179 na semana 27. A figura abaixo mostra as condições de transmissão em cada município.

% Mapa estadual
\includegraphics[width=0.8\textwidth]{../RJ/figs/Mapa_ERJ.png}
%-----------------------------------------------------------

Dos 92 municipios, 91 encontram-se em nivel verde, 0 em nivel amarelo, 0 em nivel laranja e 1 em nivel vermelho referentes a semana epidemiológica 27-2016. Para informações mais atualizadas
sobre o município do Rio de Janeiro ter acesso ao mapa interativo do estado consultar em \href{http://info.dengue.mat.br}{\textit{Info Dengue}}.


%--------------------------------------------------------------------------------
%	PRIMEIRA PAGINA : BOX - CODIGO DE CORES
%--------------------------------------------------------------------------------
\vspace{1cm}
\begin{mdframed}[style=intextbox,frametitle={}] % Sidebar box

\hypertarget{descriptivebox}{\heading{O código de Cores}{1pt}} % \hypertarget provides a label to reference using \hyperlink{label}{link text}
As cores indicam niveis de atenção
\begin{description}
\item[Verde:] temperaturas amenas, baixa incidência de casos.      
\item[Amarelo:] temperatura propícia para a população do vetor e transmissão da dengue.
\item[Laranja:] transmissão aumentada e sustentada de dengue. 
\item[Vermelho:] incidência alta de dengue, acima dos 90\% históricos.
\end{description}
\end{mdframed}

\textbf{Na semana passada:} 0 municípios em nivel amarelo, 0 em laranja e 1 em vermelho.  

\end{minipage} % End the main body - first page mini page


%----------------------------------------------------------------------------------
%	SEGUNDA PAGINA: TEXTO PRINCIPAL - REGIONAL
%----------------------------------------------------------------------------------

\begin{minipage}[t]{.66\linewidth}\hypertarget{regional}{\heading{Situação da Dengue na Regional Metropolitana I}{6pt}}\includegraphics[\width=8\textwidth]{/home/administrador/Repositorios/AlertaDengueAnalise/report/RJ/figs/MapaRJ_MetropolitanaI.png}\captionof{table}[tabregional]{Resumo das últimas seis semanas epidemiológicas na Regional Metropolitana I }\begin{center}% latex table generated in R 3.3.1 by xtable 1.8-2 package
% Tue Sep  6 10:49:32 2016
\begin{tabular}{c|ccccccc}
  \hline
SE & temperatura & tweet & casos notif & casos preditos & ICmin & ICmax & incidência \\ 
  \hline
201630 & 17 & 0 & 181 & 200 & 194 & 201 & 2 \\ 
  201631 & 18 & 0 & 98 & 111 & 105 & 112 & 1 \\ 
  201632 & 18 & 0 & 93 & 113 & 107 & 114 & 1 \\ 
  201633 & 19 & 0 & 105 & 146 & 138 & 149 & 1 \\ 
  201634 & 16 & 0 & 70 & 121 & 109 & 124 & 1 \\ 
  201635 & 19 &  & 34 & 92 & 79 & 96 & 0 \\ 
   \hline
\end{tabular}
\end{center}\small{\hyperlink{vartab}{ver descrição das variáveis}}\begin{center}\captionof{figure}{Casos notificados de dengue e Índice de menção em midia social sobre dengue na Regional Metropolitana I }\includegraphics[width=1.3\textwidth]{/home/administrador/Repositorios/AlertaDengueAnalise/report/RJ/figs/tweetRJ_MetropolitanaI.png}\end{center}\end{minipage}\hfill\begin{minipage}[t]{.30\linewidth}\begin{mdframed}[style=sidebar,frametitle={}]\textbf{\hyperlink{municips}{Municipios}}\begin{itemize}\gsquare Belford Roxo 
\gsquare Duque de Caxias 
\gsquare Itaguaí 
\gsquare Japeri 
\gsquare Magé 
\gsquare Mesquita 
\gsquare Nilópolis 
\gsquare Nova Iguaçu 
\gsquare Queimados 
\gsquare Rio de Janeiro 
\gsquare São João de Meriti 
\gsquare Seropédica \end{itemize}\BackToContents\end{mdframed}\hfill\end{minipage}\newpage%-----------------------------------------------------------------------------------%	MAIN BODY - THIRD PAGE - PARTE PRINCIPAL - Municipio
%-----------------------------------------------------------------------------------
 \begin{minipage}[t]{0.6\linewidth} % Mini page taking up 100% of the actual page
\hypertarget{mun}{\heading{Situação da Dengue na Cidade do Rio de Janeiro: Mapa}{6pt}} % \hypertarget 

%A figura abaixo mostra a série histórica do município: (TOPO) Série temporal de casos suspeitos de dengue e série temporal de dengue mencionada em rede social; (MEIO) Perfil da temperatura; (BAIXO) histórico do alerta. 

\includegraphics[width=0.95\textwidth]{../Rio_de_Janeiro/mapaRio.png}

\small{Veja o mapa interativo em http://alerta.dengue.mat.br/rio}

\vspace{2cm}

\captionof{table}{Resumo das últimas seis semanas epidemiológicas}
\begin{center}
% latex table generated in R 3.3.0 by xtable 1.8-2 package
% Tue Aug 30 11:47:58 2016
\begin{tabular}{c|ccccccc}
  \hline
se & casos & casos.estimados & ICmin & ICmax & inc & tweet & tmin \\ 
  \hline
201629 & 165 & 165 & 165 & 165 & 3 & 8 & 18 \\ 
  201630 & 154 & 154 & 154 & 154 & 2 & 9 & 20 \\ 
  201631 & 76 & 76 & 76 & 76 & 1 & 41 & 20 \\ 
  201632 & 68 & 68 & 68 & 68 & 1 & 26 & 19 \\ 
  201633 & 63 & 63 & 63 & 63 & 1 & 6 & 21 \\ 
  201634 & 37 & 53 & 45 & 56 & 1 & 2 & 18 \\ 
   \hline
\end{tabular}

\end{center}

\BackToContents % Link back to the contents of the newsletter
\end{minipage}
\begin{minipage}[t]{.26\linewidth} 
\begin{mdframed}[style=sidebar,frametitle={}] 

%%--------------------------------------------------------------------------
%%      TERCEIRA PAGINA: CAIXA LATERAL - INDICE APS
%%-------------------------------------------------------------------------

\textbf{Áreas Programáticas da Saúde} 

\begin{itemize}\gsquare APS  1.0 
\gsquare APS  2.1 
\gsquare APS  2.2 
\gsquare APS  3.1 
\gsquare APS  3.2 
\gsquare APS  3.3 
\gsquare APS  4.0 
\gsquare APS  5.1 
\gsquare APS  5.2 
\gsquare APS  5.3 \end{itemize}
\end{mdframed} 
\end{minipage}
\newpage

%-----------------------------------------------------------------------------------%	
% MAIN BODY - FOURTH PAGE - Situação da cidade
%-----------------------------------------------------------------------------------

\hypertarget{municipio}{\heading{Situação da Dengue na Cidade do Rio de Janeiro: Séries Históricas}{6pt}} % \hypertarget 

\vspace{2cm}
\includegraphics[width=0.8\textwidth]{../Rio_de_Janeiro/figcidade.png}

A linha tracejada verde indica o limiar pré-epidêmico; a linha tracejada vermelha indica o limiar de atividade alta (acima do qual é acionado o alerta vermelho). 

\BackToContents % Link back to the contents of the newsletter


\newpage

%-----------------------------------------------------------------------------------%	
% MAIN BODY - FOURTH PAGE - Situação por APS - graficos
%-----------------------------------------------------------------------------------

\hypertarget{aps}{\heading{Situação nas Áreas Programáticas de Saúde}{6pt}} % \hypertarget 

\vspace{2cm}
\includegraphics[width=0.8\textwidth]{../Rio_de_Janeiro/figaps1.png}

(cont.)

\newpage

\heading{Situação nas Áreas Programáticas de Saúde (cont.)}{6pt}} 

\vspace{2cm}
\includegraphics[width=0.8\textwidth]{../Rio_de_Janeiro/figaps2.png}

\newpage

%-----------------------------------------------------------------------------------%	
% MAIN BODY - FIFTH PAGE - Situação por APS - tabelas
%-----------------------------------------------------------------------------------

\heading{Situação nas Áreas Programáticas de Saúde: Tabelas}{6pt}

\captionof{table}[tabregional]{Resumo das últimas seis semanas epidemiológicas nas Áreas Programáticas de Saúde}\begin{center}\captionof{table}{APS 1.0 }
% latex table generated in R 3.3.0 by xtable 1.8-2 package
% Tue Aug 23 13:17:47 2016
\begin{tabular}{c|ccccccc}
  \hline
se & casos & casos\_est & tmin & rt & p\_rt1 & inc & nivel \\ 
  \hline
201630 & 2 & 2 & 20 & 0 & 0 & 1 & 1 \\ 
  201631 & 3 & 3 & 20 & 1 & 0 & 1 & 1 \\ 
  201632 & 0 & 0 & 19 & 0 & 0 & 0 & 1 \\ 
  201633 & 7 & 10 & 21 & 4 & 1 & 4 & 1 \\ 
   \hline
\end{tabular}

\captionof{table}{APS 2.1 }
% latex table generated in R 3.3.1 by xtable 1.8-2 package
% Tue Sep  6 12:23:33 2016
\begin{tabular}{c|ccccccc}
  \hline
se & casos & casos\_est & tmin & rt & p\_rt1 & inc & nivel \\ 
  \hline
201632 & 1 & 1 & 19 & 0 & 0 & 0 & 1 \\ 
  201633 & 6 & 6 & 21 & 2 & 1 & 1 & 1 \\ 
  201634 & 4 & 4 & 18 & 1 & 1 & 1 & 1 \\ 
  201635 & 0 & 0 & 21 & 0 & 0 & 0 & 1 \\ 
   \hline
\end{tabular}

\captionof{table}{APS 2.2 }
% latex table generated in R 3.3.1 by xtable 1.8-2 package
% Tue Sep 20 21:34:20 2016
\begin{tabular}{c|ccccccc}
  \hline
se & casos & casos\_est & tmin & rt & p\_rt1 & inc & nivel \\ 
  \hline
201634 & 1 & 1 & 18 & 1 & 0 & 0 & 1 \\ 
  201635 & 1 & 1 & 21 & 1 & 0 & 0 & 1 \\ 
  201636 & 11 & 11 & 21 & 6 & 1 & 3 & 1 \\ 
  201637 & 1 & 1 & 22 & 0 & 0 & 0 & 1 \\ 
   \hline
\end{tabular}

\captionof{table}{APS 3.1 }
% latex table generated in R 3.3.0 by xtable 1.8-2 package
% Tue Aug 23 13:17:47 2016
\begin{tabular}{c|ccccccc}
  \hline
se & casos & casos\_est & tmin & rt & p\_rt1 & inc & nivel \\ 
  \hline
201630 & 28 & 28 & 20 & 1 & 0 & 4 & 1 \\ 
  201631 & 12 & 12 & 20 & 0 & 0 & 2 & 1 \\ 
  201632 & 17 & 17 & 19 & 1 & 0 & 2 & 1 \\ 
  201633 & 13 & 20 & 21 & 1 & 1 & 3 & 1 \\ 
   \hline
\end{tabular}

\captionof{table}{APS 3.2 }
% latex table generated in R 3.3.1 by xtable 1.8-2 package
% Tue Sep 20 21:34:20 2016
\begin{tabular}{c|ccccccc}
  \hline
se & casos & casos\_est & tmin & rt & p\_rt1 & inc & nivel \\ 
  \hline
201634 & 2 & 2 & 18 & 0 & 0 & 0 & 1 \\ 
  201635 & 2 & 2 & 21 & 0 & 0 & 0 & 1 \\ 
  201636 & 0 & 0 & 21 & 0 & 0 & 0 & 1 \\ 
  201637 & 2 & 2 & 22 & 1 & 1 & 0 & 1 \\ 
   \hline
\end{tabular}
\end{center}\small{\hyperlink{vartab}{ver descrição das variáveis}}(cont.)
\newpage

\heading{Situação nas Áreas Programáticas de Saúde: Tabelas (cont.)}{6pt}

\captionof{table}[tabregional]{Resumo das últimas seis semanas epidemiológicas na Regional Centro Sul }\begin{center}\captionof{table}{APS 3.3 }
% latex table generated in R 3.3.0 by xtable 1.8-2 package
% Tue Aug 23 13:17:47 2016
\begin{tabular}{c|ccccccc}
  \hline
se & casos & casos\_est & tmin & rt & p\_rt1 & inc & nivel \\ 
  \hline
201630 & 27 & 27 & 20 & 1 & 0 & 3 & 1 \\ 
  201631 & 10 & 10 & 20 & 0 & 0 & 1 & 1 \\ 
  201632 & 12 & 12 & 19 & 1 & 0 & 1 & 1 \\ 
  201633 & 3 & 4 & 21 & 0 & 0 & 0 & 1 \\ 
   \hline
\end{tabular}

\captionof{table}{APS 4.0 }
% latex table generated in R 3.3.0 by xtable 1.8-2 package
% Tue Aug 23 13:17:47 2016
\begin{tabular}{c|ccccccc}
  \hline
se & casos & casos\_est & tmin & rt & p\_rt1 & inc & nivel \\ 
  \hline
201630 & 16 & 16 & 20 & 1 & 0 & 2 & 1 \\ 
  201631 & 10 & 10 & 20 & 0 & 0 & 1 & 1 \\ 
  201632 & 6 & 6 & 19 & 0 & 0 & 1 & 1 \\ 
  201633 & 1 & 1 & 21 & 0 & 0 & 0 & 1 \\ 
   \hline
\end{tabular}

\captionof{table}{APS 5.1 }
% latex table generated in R 3.3.0 by xtable 1.8-2 package
% Tue Aug 23 13:17:47 2016
\begin{tabular}{c|ccccccc}
  \hline
se & casos & casos\_est & tmin & rt & p\_rt1 & inc & nivel \\ 
  \hline
201630 & 13 & 13 & 20 & 0 & 0 & 2 & 1 \\ 
  201631 & 8 & 8 & 20 & 0 & 0 & 1 & 1 \\ 
  201632 & 11 & 11 & 19 & 1 & 0 & 2 & 1 \\ 
  201633 & 5 & 7 & 21 & 1 & 0 & 1 & 1 \\ 
   \hline
\end{tabular}

\captionof{table}{APS 5.2 }
% latex table generated in R 3.3.0 by xtable 1.8-2 package
% Tue Aug 23 13:17:47 2016
\begin{tabular}{c|ccccccc}
  \hline
se & casos & casos\_est & tmin & rt & p\_rt1 & inc & nivel \\ 
  \hline
201630 & 10 & 10 & 20 & 0 & 0 & 2 & 1 \\ 
  201631 & 16 & 16 & 20 & 1 & 0 & 2 & 1 \\ 
  201632 & 4 & 4 & 19 & 0 & 0 & 1 & 1 \\ 
  201633 & 2 & 2 & 21 & 0 & 0 & 0 & 1 \\ 
   \hline
\end{tabular}

\captionof{table}{APS 5.3 }
% latex table generated in R 3.3.1 by xtable 1.8-2 package
% Fri Sep  9 10:20:38 2016
\begin{tabular}{c|ccccccc}
  \hline
se & casos & casos\_est & tmin & rt & p\_rt1 & inc & nivel \\ 
  \hline
201631 & 3 & 3 & 20 & 0 & 0 & 1 & 1 \\ 
  201632 & 4 & 4 & 19 & 0 & 0 & 1 & 1 \\ 
  201633 & 3 & 3 & 21 & 1 & 0 & 1 & 1 \\ 
  201634 & 5 & 5 & 18 & 1 & 1 & 1 & 1 \\ 
   \hline
\end{tabular}
\end{center}\small{\hyperlink{vartab}{ver descrição das variáveis}}
%---------------------------------------------------------------------------------
%	Variáves nas Tabelas, Créditos e Contato
%---------------------------------------------------------------------------------

\begin{minipage}[t]{1\linewidth} 

\hypertarget{vartab}{\heading{Lista das variáveis apresentadas nas tabelas:}}

\begin{description}
\item [SE =] semana epidemiológica
\item [tweet =] número de tweets indicativos de casos de dengue na cidade
\item [temperatura =] média das temperaturas mínimas da semana
\item [casos notif =] casos notificados de dengue 
\item [casos preditos =] número de casos estimados após correção pelo atraso de notificação
\item [ICmin =] número mínimo de casos estimados (IC 95\%)
\item [ICmax =] número máximo de casos estimados (IC 95\%)
\item [Rt] número reprodutivo efetivo ($>$ 1 indica aumento de casos transmissão)
\item [p(Rt1) =] probabilidade do número reprodutivo ser maior que 1 ($>0.95$ indica aumento significativo de casos)
\item [inc =] incidência por 100.000 habitantes
\item [Nivel =] cor do alerta (verde, amarelo, laranja, vermelho)
\end{description}

\hypertarget{notas}{\heading{Notas}}

\begin{itemize}
\item Os dados do sinan mais recentes ainda não foram totalmente digitados. Estimamos o número esperado de casos notificados considerando o tempo até os casos serem digitados.
\item Os dados de tweets são gerados pelo Observatório de Dengue (UFMG). Os tweets são processados para exclusão de informes e outros temas relacionados a dengue.
\item Algumas vezes, os casos da última semana ainda não estao disponíveis, nesse caso, usa-se uma estimação com base na tendência de variação da série.
\end{itemize}

\hypertarget{creditos}{\heading{Créditos}}

Este é um projeto desenvolvido com apoio da SVS/MS em parceria com:

\begin{itemize}
\item Programa de Computação Científica, Fundação Oswaldo Cruz, Rio de Janeiro.
\item Escola de Matemática Aplicada, Fundação Getúlio Vargas.
\item Secretarias do Estado e Município do Rio de Janeiro.
\item Observatório de Dengue da UFMG
\item Secretaria Estadual de Saúde do Paraná.
\end{itemize}

      \BackToContents % Link back to the contents of the newsletter

\vspace{1cm}

\hline
Para mais detalhes sobre o sistema de alerta InfoDengue, consultar: \url{http://info.dengue.mat.br}\\

\textbf{Contato}: \href{alerta\_dengue@fiocruz.br}{\nolinkurl{alerta\_dengue@fiocruz.br} }
\end{minipage} % fim da pagina de creditos

\end{document} 
